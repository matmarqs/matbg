%%%%%%%%%%%%%%%%%%%%%%%%%%%%%%%% COMMENT THIS TO COMPILE main.tex %%%%%%%%%%%%%%%%%%%%%%%%%%%%%%%%
\documentclass[a4paper,12pt]{report}
\usepackage[english]{babel}
\usepackage[left=2cm,right=2cm,top=2cm,bottom=2cm]{geometry}
%\usepackage{mathtools}
\usepackage{amsthm}     % for definitions and theorems
\usepackage[many]{tcolorbox}    % boxes around definitions and theorems
%\usepackage{amsmath}
%\usepackage{nccmath}
\usepackage{amssymb}    % \ltimes
\usepackage{etoolbox}   % for start of Chapter
%\usepackage{amsfonts}
\usepackage{physics}    % for all Physics related
%\usepackage{dsfont}
%\usepackage{mathrsfs}

\usepackage{titling}
\usepackage{indentfirst}

\usepackage{bm}
\usepackage[dvipsnames]{xcolor}
\usepackage{cancel}

\usepackage{xurl}
\usepackage[colorlinks=true]{hyperref}

\usepackage{float}
\usepackage{graphicx}
\usepackage{subcaption}
%\usepackage{tikz}

\usepackage{ctable}     % tabelas
\renewcommand{\P}{\phantom{+}}  % empty space to indent things
\usepackage{multirow}
\usepackage{tabulary}

%%%%%%%%%%%%%%%%%%%%%%%%%%%%%%%%%%%%%%%%%%%%%%%%%%%

\newcommand{\eps}{\epsilon}
\newcommand{\vphi}{\varphi}
\newcommand{\cte}{\text{cte}}

\newcommand{\N}{{\mathbb{N}}}
\newcommand{\Z}{{\mathbb{Z}}}
%\newcommand{\Q}{{\mathbb{Q}}}
\newcommand{\C}{{\mathbb{C}}}
\renewcommand{\S}{{\hat{S}}}
%\renewcommand{\H}{\s{H}}

\renewcommand{\a}{{\vb{a}}}
\renewcommand{\b}{{\vb{b}}}
\renewcommand{\d}{{\dagger}}
\newcommand{\up}{{\uparrow}}
\newcommand{\down}{{\downarrow}}
\newcommand{\hc}{{\text{h.c.}}}

\newcommand{\ihat}{\bm{\hat{\imath}}}
\newcommand{\jhat}{\bm{\hat{\jmath}}}
\newcommand{\khat}{\bm{\hat{k}}}

\newcommand{\0}{{\vb{0}}}
%\newcommand{\1}{\mathds{1}}
\newcommand{\E}{{\vb{E}}}
\newcommand{\B}{{\vb{B}}}
\renewcommand{\u}{{\vb{u}}}
\renewcommand{\v}{{\vb{v}}}
\renewcommand{\r}{{\vb{r}}}
\newcommand{\R}{{\vb{R}}}
\newcommand{\Q}{{\vb{Q}}}
\newcommand{\G}{{\vb{G}}}
\newcommand{\g}{{\vb{g}}}
\renewcommand{\k}{{\vb{k}}}
\newcommand{\K}{{\vb{K}}}
\newcommand{\p}{{\vb{p}}}
\newcommand{\q}{{\vb{q}}}
\newcommand{\F}{{\vb{F}}}
\renewcommand{\t}{{\vb{t}}}
\newcommand{\vtau}{{\bm{\tau}}}
\newcommand{\vdelta}{{\bm{\delta}}}

\newcommand{\s}{\sigma}
%\newcommand{\prodint}[2]{\left\langle #1 , #2 \right\rangle}
\newcommand{\cc}[1]{\overline{#1}}
\newcommand{\Eval}[3]{\eval{\left( #1 \right)}_{#2}^{#3}}
\newcommand{\sg}[2]{\{ #1 \mid #2 \}}

\newcommand{\unit}[1]{\; \mathrm{#1}}

\newcommand{\n}{\medskip}
\newcommand{\e}{\quad \mathrm{and} \quad}
\newcommand{\ou}{\quad \mathrm{or} \quad}
\newcommand{\virg}{\, , \;}
\newcommand{\ptodo}{\forall \,}
\renewcommand{\implies}{\; \Rightarrow \;}
%\newcommand{\eqname}[1]{\tag*{#1}} % Tag equation with name

\setlength{\droptitle}{-7em}

\makeatletter
\patchcmd{\chapter}{\if@openright\cleardoublepage\else\clearpage\fi}{}{}{}  % start 'Chapter' at the same page. needs package etoolbox
\makeatother

%% Theorems, definitions, proofs
\theoremstyle{definition}

\newtheorem{definition}{Definition}[section]
\tcolorboxenvironment{definition}{
  colback=blue!5!white,
  boxrule=0pt,
  boxsep=1pt,
  left=2pt,right=2pt,top=2pt,bottom=2pt,
  oversize=2pt,
  sharp corners,
  before skip=\topsep,
  after skip=\topsep,
}

\newtheorem{theorem}{Theorem}[section]
\tcolorboxenvironment{theorem}{
  colback=blue!5!white,
  boxrule=0pt,
  boxsep=1pt,
  left=2pt,right=2pt,top=2pt,bottom=2pt,
  oversize=2pt,
  sharp corners,
  before skip=\topsep,
  after skip=\topsep,
}

\begin{document}
%%%%%%%%%%%%%%%%%%%%%%%%%%%%%%%% COMMENT THIS TO COMPILE main.tex %%%%%%%%%%%%%%%%%%%%%%%%%%%%%%%%


%%%%%%%%%%%%%%%%%%%%%%%%%%%%%%%%%%%%%%%%%%%%%%%%%%%%%%%%%%%%%%%%%%%%%%%%%%%%%%%%%%%%%%%%%%%%%%%%%%
\chapter{Emmergent symmetries and Wannier obstruction in TBG} \label{ch:emmergent_symm_wannier_obstruction}
%%%%%%%%%%%%%%%%%%%%%%%%%%%%%%%%%%%%%%%%%%%%%%%%%%%%%%%%%%%%%%%%%%%%%%%%%%%%%%%%%%%%%%%%%%%%%%%%%%

%\textbf{THE TEXT BELOW IS PLACEHOLDER. IT WAS COPIED FROM PREVIOUS REPORTS}
%
%Regarding (a), in Section \ref{sec:tbg}, we present our progress in understanding the MATBG system by studying its geometry \cite{handbook2019}, the Bistritzer-MacDonald (BM) model \cite{macdonald2011}, and its symmetry properties \cite{thesis_rennella}. Symmetry analysis is a prerequisite for comprehending the Wannier obstruction phenomenon \cite{zou2018}, which arises from the non-trivial topology of the electronic bands. This phenomenon refers to the impossibility of representing the flat bands using exponentially localized Wannier functions that preserve all symmetries of the band structure. In contrast, the THF model addresses this issue by providing a fully symmetric framework that reformulates and maps the interacting MATBG as a topological heavy fermion system \cite{topoheavyfermion2022}.
%
%We initiated our exploration of the MATBG theory with reference to \cite{handbook2019}, aiming primarily to comprehend the Moiré pattern geometry and the Bistritzer-MacDonald continuous model \cite{macdonald2011}. Subsequently, our focus shifted to analyzing the system's symmetries \cite{thesis_rennella}, which are pivotal, with implications such as the Wannier obstruction \cite{zou2018}. In Section \ref{sec:tbg}, we elucidate our studies.
%
%By implementing the BM model \cite{macdonald2011}, when we twist the two graphene layers (from the initial AB stacking) by a discrete set of very specific angles, with the first being $\theta \approx 1.05^\circ$, the bands should become very flat around the point $K$, which corresponds to the low-energy excitations of the system.
%
%\textbf{THE TEXT ABOVE IS PLACEHOLDER. IT WAS COPIED FROM PREVIOUS REPORTS}

%%%%%%%%%%%%%%%%%%%%%%%%%%%%%%%%%%%%%%%%%%%%%%%%%%%%%%%%%%%%%%%%%%%%%%%%%%%%%%%%%%%%%%%%%%%%%%%%%%
\section{Commensurate structures} \label{sec:tbg_geom}
%%%%%%%%%%%%%%%%%%%%%%%%%%%%%%%%%%%%%%%%%%%%%%%%%%%%%%%%%%%%%%%%%%%%%%%%%%%%%%%%%%%%%%%%%%%%%%%%%%

Twisted bilayer graphene (TBG) features two graphene layers rotated relative to each other by an angle \( \theta \). This rotation gives rise to a Moiré pattern, formed by the interference of the two lattices. The pattern can be either \textit{commensurate}, exhibiting translational symmetry and forming a periodic superlattice, or \textit{incommensurate}, lacking translational symmetry and periodicity. Figure \ref{fig:moireD6} illustrates a commensurate Moiré pattern with \( D_6 \) point group symmetry, where the rotation center lies at the center of a hexagon. The AA, AB, and BA regions in the pattern represent areas where \( A \) or \( B \) carbon atoms from the layers align with each other.

The system can generally exhibit different symmetry groups depending on the choice of the origin for the rotation by \( \theta \). However, the largest possible point group is \( D_6 \), due to the honeycomb Bravais lattice of the two layers. In the commensurate case, the primitive lattice vectors \( \vb{L}_1 \) and \( \vb{L}_2 \) are defined as the least common multiples of the unit vectors of the monolayers. Figure \ref{fig:latvec} provides an example for \( \theta = 21.8^\circ \).
\begin{figure}[H]
\centering
\includegraphics[height=.32\columnwidth]{fig/moireD6.png}
\caption{Commensurate $D_6$-type structure obtained by twisting the two layers from the initial AA stacking configuration, with the origin chosen as the center of the hexagons. Figure taken from \cite{thesis_rennella}.}
\label{fig:moireD6}
\end{figure}

%As drawn in Figure \ref{fig:latvec}, we have
%\begin{align}
%\label{eq:scalarprods1}
%\vb{a}_1^{(1)} \vdot \vb{a}_2^{(1)} &= a^2 \cos(60^\circ) = a^2/2; \\
%\label{eq:scalarprods2}
%\vb{a}_1^{(1)} \vdot \vb{a}_1^{(2)} &= a^2 \cos\theta; \\
%\label{eq:scalarprods3}
%\vb{a}_1^{(1)} \vdot \vb{a}_2^{(2)} &= a^2 \cos(60^\circ + \theta); \\
%\label{eq:scalarprods4}
%\vb{a}_1^{(2)} \vdot \vb{a}_2^{(1)} &= a^2 \cos(60^\circ - \theta).
%\end{align}

The superlattice vectors $\vb{L}_1$, $\vb{L}_2$ are related by a $60^\circ$ rotation. In general, because $\vb{L}_1$ belongs to the lattices of both layers, it can be parameterized by integers $n_1^{(1)}, n_2^{(1)}, n_1^{(2)}, n_2^{(2)}$ as
\begin{equation} \label{eq:L1}
\vb{L}_1 = n_1^{(1)}\vb{a}_1^{(1)} + n_2^{(1)}\vb{a}_2^{(1)} = n_1^{(2)}\vb{a}_1^{(2)} + n_2^{(2)}\vb{a}_2^{(2)},
\end{equation}
where $\vb{a}_{1,2}^{(\ell)}$ are the unit vectors of layer $\ell = 1,2$.

The resulting crystal will exhibit exact lattice translational symmetry with the superlattice vectors \( \vb{L}_1 \) and \( \vb{L}_2 \) only if \( \theta \) satisfies the commensurability condition \cite{thesis_rennella, zou2018}, given by integers \( m \) and \( r \):

\begin{equation} \label{eq:costheta}
\cos\theta(m,r) = \frac{3m^2 + 3mr + r^2/2}{3m^2 + 3mr + r^2} = \frac{3 + 3 \qty(\frac{r}{m}) + \frac{1}{2} \qty(\frac{r}{m})^2}{3 + 3 \qty(\frac{r}{m}) + \qty(\frac{r}{m})^2}.
\end{equation}

We point out that \( \theta(m,r) \) is a function of only the ratio \( r/m \) and is monotonic within the range \( 0 \leq \theta \leq 60^\circ \). Therefore, all commensurate angles can be uniquely determined by pairs of coprime integers \( (m, r) \), where \( \gcd(m, r) = 1 \).

There are two types of structures, classified as Type I and Type II \cite{zou2018} (or SE-odd and SE-even, respectively, with SE standing for sublattice exchange \cite{continuum_model_lopesdossantos2012}), each characterized by distinct relationships between the superlattice vectors \(\vb{L}_1\) and \(\vb{L}_2\) and the primitive vectors of the layers. These relationships can be expressed in terms of the primitive vectors of the first layer, \(\a_1^{(1)}\) and \(\a_2^{(1)}\), as follows:
\begin{itemize}

\item \textit{Type I (SE-odd)}, when $\gcd(r, 3) = 1$:
\begin{equation} \label{eq:type-I-L1L2}
\begin{pmatrix}
\vb{L}_1 \\ \vb{L}_2
\end{pmatrix} =
%\begin{pmatrix}
%1 & -1 \\
%1 & 0
%\end{pmatrix}
%\begin{pmatrix}
%m & 2m+r \\
%-(m+r) & m
%\end{pmatrix}
%\begin{pmatrix}
%1 & 0 \\
%-1 & 1
%\end{pmatrix}
%\begin{pmatrix}
%\a_1^* \\ \a_2^*
%\end{pmatrix}
%=
\begin{pmatrix}
m & m+r \\
-(m+r) & 2m+r
\end{pmatrix}
\begin{pmatrix}
\a_1^{(1)} \\ \a_2^{(2)}
\end{pmatrix}.
\end{equation}

\item \textit{Type II (SE-even)} when $\gcd(r,3) = 3$:

\begin{equation} \label{eq:type-II-L1L2}
\begin{pmatrix}
\vb{L}_1 \\ \vb{L}_2
\end{pmatrix} =
%\begin{pmatrix}
%1 & -1 \\
%1 & 0
%\end{pmatrix}
%\begin{pmatrix}
%m+\frac{r}{3} & m+\frac{2r}{3} \\
%-\frac{r}{3} & m+\frac{r}{3}
%\end{pmatrix}
%\begin{pmatrix}
%1 & 0 \\
%-1 & 1
%\end{pmatrix}
%\begin{pmatrix}
%\a_1^{*} \\ \a_2^{*}
%\end{pmatrix}
%=
\begin{pmatrix}
m+\frac{r}{3} & \frac{r}{3} \\
-\frac{r}{3} & m+\frac{2r}{3}
\end{pmatrix}
\begin{pmatrix}
\a_1^{(1)} \\ \a_2^{(2)}
\end{pmatrix}.
\end{equation}

\end{itemize}

For both Type I and Type II structures, the moiré lattice constant \( L(m, r) = \abs{\vb{L}_1} = \abs{\vb{L}_2} \) is determined by the following formula:
\begin{equation} \label{eq:commensurate-constant}
L(m, r) = a \sqrt{\frac{3m^2 + 3mr + r^2}{\gcd(r, 3)}},
\end{equation}
where \(a = 0.246 \unit{nm}\) is the lattice constant of the monolayer, and \(\gcd(r, 3)\) accounts for the sublattice exchange parity.

The area of the unit cell of the moiré superlattice is given by
\begin{equation} \label{eq:superlattice_area_unitcell}
A_{\text{moiré}} = \frac{\sqrt{3}}{2} L^2 = \frac{3m^2 + 3mr + r^2}{\gcd(r,3)} \, A_1,
\end{equation}
where \( A_1 = \frac{\sqrt{3}}{2} \, a^2 \) represents the area of the unit cell of monolayer graphene. As the unit cell area of the moiré superlattice increases, its corresponding Brillouin Zone shrinks, forming what is referred to as the mini Brillouin Zone (mBZ). The area of the mBZ is given by
\begin{equation} \label{eq:bz-volume}
\Omega_{\text{mBZ}} = \frac{(2\pi)^2}{A_{\text{moiré}}} = \frac{\gcd(r,3)}{3m^2 + 3mr + r^2} \, \Omega_1,
\end{equation}
where \( \Omega_1 = \frac{(2\pi)^2}{A_1} \) is the area of the monolayer's BZ.

For instance, when \( (m, r) = (1, 1) \), we have \( \gcd(r,3) = 1 \) (Type I structure) and a corresponding twist angle \( \theta(1,1) = 21.8^\circ \). In this case, \( 3m^2 + 3mr + r^2 = 7 \), meaning that the monolayer BZ accommodates 7 mBZs within its area, as illustrated in Figure \ref{fig:bzminibz}.

\begin{figure}[H]
\centering
\begin{subfigure}{.5\textwidth}
  \centering
  \includegraphics[height=.6\linewidth]{fig/latvec.png}
  \caption{}
  \label{fig:latvec}
\end{subfigure}%
\begin{subfigure}{.5\textwidth}
  \centering
  \includegraphics[height=.6\linewidth]{fig/bzminibz.png}
  \caption{}
  \label{fig:bzminibz}
\end{subfigure}
\caption{(a) Commensurate structure and lattice vectors $\vb{L}_1$, $\vb{L}_2$. (b) Monolayers (red and green) and mini BZ's with $(m,r) = (1,1)$ and $\theta = 21.8^\circ$. Figures taken from \cite{koshino2012}. %\textbf{SEPARAR AS FIGURAS}
}
\label{fig:geometry}
\end{figure}

The distinct relationships between the superlattice vectors for Type I and Type II structures lead to different folding patterns for the Dirac points \(K^{(1)}\) and \(K^{(2)}\) of the monolayers into the mBZ. In Type I structures, the Dirac points \(K^{(1)}\) and \(K^{(2)}\) fold into different time-reversal types of Dirac points in the mBZ, with \(K^{(1)}\) folding to \(K_m'\) and \(K^{(2)}\) folding to \(K_m\), as illustrated in Figure \ref{fig:bzminibz}. In contrast, for Type II structures, \(K^{(1)}\) and \(K^{(2)}\) fold into the same type of Dirac point, \(K_m\), in the mBZ. For a more detailed discussion of these folding patterns, refer to \cite{zou2018}.


We emphasize that the commensuration condition for TBG depends only on the twist angle $\theta$, not the twisting center. As long as the twist angle satisfies Equation \ref{eq:costheta}, the bilayer retains exact translational symmetry, even if the twisting center is a generic point where no two carbon atoms align perfectly. The Moiré lattice vectors $\vb{L}_1$ and $\vb{L}_2$, lattice constant \( L(m, r) \), and momentum mapping between the microscopic and Moiré Brillouin zones (BZs) depend solely on the twist angle and not the twisting center. However, the twisting center determines the exact point-group symmetries of the commensurate lattice. If the twisting center is a generic point, translations are the only exact spatial symmetries. If it is the center of a hexagon as in Figure \ref{fig:moireD6}, the twisted bilayer graphene (TBG) inherits the six-fold rotational symmetry \( C_6 \) of the monolayers.

From the discussion above, one would conclude that a commensurate TBG system should have a translational symmetry with lattice constant $L(\theta)$ given by Eq. \eqref{eq:commensurate-constant}. However, as discussed in \cite{zou2018}, scanning tunneling microscopy (STM) experiments actually observe an effective moiré lattice constant
\begin{equation} \label{eq:STM-constant}
L'(\theta) = \frac{a}{2 \sin(\theta/2)} \leq \frac{r}{\sqrt{\gcd(r,3)}} \cdot \frac{a}{2 \sin(\theta/2)} = L(\theta).
\end{equation}

This can be interpreted as TBG exhibiting an approximate translational symmetry with superlattice constant $L'(\theta)$, despite optionally possessing an exact translational symmetry with $L(\theta)$ in the commensurate case.

The Bistritzer-MacDonald model (also known as the ``continuum theory'') incorporates translational symmetry by a lattice constant $L'(\theta)$ and other emergent symmetries observed in TBG experiments. This theory describes the system at every twist angle $\theta \lesssim 10^\circ$, including incommensurate ones \cite{continuum_model_castroneto2007, macdonald2011}. It predicts universal features, such as Dirac crossings between valence and conduction bands in each valley, which match tight-binding calculations for commensurate structures and experimental data for larger twist angles, where correlation effects are less significant \cite{zou2018}. However, the theory underestimates the gap between the nearly flat bands and the other bands. This discrepancy can be addressed by incorporating effects like lattice relaxation and electron interactions, which can be phenomenologically accounted for by adjusting the model's parameters \cite{koshinohubbard2018}.

The continuum theory includes additional symmetries, such as \(C_6 T\) rotation and valley \(U_v(1)\) symmetries, which, while not fully present in commensurate structures, are essential for protecting Dirac points in valley-filtered bands. The \(C_2 T = (C_6 T)^3\) symmetry prevents the Dirac points from opening a gap. Although these symmetries are approximations, they remain effective in small-angle TBG experiments, making the differences between commensurate and incommensurate structures negligible.

A key feature of the band structure is the inability to construct well-localized Wannier functions that respect all symmetry operations. This issue arises because some symmetries in the continuum theory, such as \(C_2 T\) and \(U_v(1)\), are not exact microscopic symmetries, yet they are essential for preserving Dirac crossings.

In conclusion, it is essential to study systems with translational, \(C_6 T\), and \(U_v(1)\) emergent symmetries, even though fully implementing these symmetries leads to a Wannier obstruction in the band structure. In Section \ref{sec:BM-model}, we derive the continuum model, and in Section \ref{sec:wannier_obstruction}, we address the related Wannier obstruction.

\textbf{FALAR QUE TIGHT-BINDING E DMFT SÃO MUITO HEAVY. E CONTINUUM MODEL É LIGHT}

\textbf{MENCIONAR LATTICE RELAXATION?}

%%%%%%%%%%%%%%%%%%%%%%%%%%%%%%%%%%%%%%%%%%%%%%%%%%%%%%%%%%%%%%%%%%%%%%%%%%%%%%%%%%%%%%%%%%%%%%%%%%
\section{Bistritzer-MacDonald model} \label{sec:BM-model}
%%%%%%%%%%%%%%%%%%%%%%%%%%%%%%%%%%%%%%%%%%%%%%%%%%%%%%%%%%%%%%%%%%%%%%%%%%%%%%%%%%%%%%%%%%%%%%%%%%

\textbf{Qual o objetivo desta seção? Com certeza chegar na hamiltoniana do BM-Bernevig para resolver numericamente.}

\textbf{OK! Então vou seguir a tese do Angeli, que é a referência mais didática a fim de chegar lá.}

\textbf{Sim! Mas tenho que seguir a convenção do artigo de THF.}

%%%%%%%%%%%%%%%%%%%%%%%%%%%%%%%%%%%%%%%%%%%%%%%%%%%%%%%%%%%%%%%%%%%%%%%%%%%%%%%%%%%%%%%%%%%%%%%%%%
\subsection{Monolayers}
%%%%%%%%%%%%%%%%%%%%%%%%%%%%%%%%%%%%%%%%%%%%%%%%%%%%%%%%%%%%%%%%%%%%%%%%%%%%%%%%%%%%%%%%%%%%%%%%%%

The single-layer Dirac Hamiltonian, for momentum $\k$ around $K$ or $K' = -K$, is
\begin{equation} \label{eq:}
H_\k^\eta = -\nu (\k - \zeta K) \vdot \qty(\zeta \s_x, \s_y),
\end{equation}
where $\zeta = \pm 1$ is the valley index ($\zeta = +1$ for $K$ and $\zeta = -1$ for $K = -K$).

The Dirac nodes of each monolayer are $K^{(\ell)} = R(\theta_\ell) K$.

%%%%%%%%%%%%%%%%%%%%%%%%%%%%%%%%%%%%%%%%%%%%%%%%%%%%%%%%%%%%%%%%%%%%%%%%%%%%%%%%%%%%%%%%%%%%%%%%%%
\subsection{Hybridization}
%%%%%%%%%%%%%%%%%%%%%%%%%%%%%%%%%%%%%%%%%%%%%%%%%%%%%%%%%%%%%%%%%%%%%%%%%%%%%%%%%%%%%%%%%%%%%%%%%%

The atomic 2D positions for each layer $\ell$ are given by
\begin{equation} \label{eq:position-atoms-tbg}
\r = n_1 \a_{1}^{(\ell)} + n_2 \a_{2}^{(\ell)} + \vtau_{\alpha}^{(\ell)},
\end{equation}
where $\alpha = A, B$ indexes the site type, and $\tau_{\alpha}^{(\ell)}$ are the basis vectors. The lattice vectors $\a_{j}^{(\ell)}$ are obtained by rotating the unrotated vectors $\a_j$ by an angle $\theta_\ell$ using the rotation matrix $R_{\theta_\ell}$. In our coordinate system, $\a_1 = a(1,0)$ and $\a_2 = a\qty(\frac{1}{2}, \frac{\sqrt{3}}{2})$. The rotation matrix is defined as
\begin{equation} \label{eq:rotation-matrix}
R_\theta =
\begin{pmatrix}
\cos\theta & -\sin\theta \\
\sin\theta & \cos\theta
\end{pmatrix}.
\end{equation}
We choose the reference frame such that $\theta_1 = -\theta/2$ and $\theta_2 = +\theta/2$, with the AA-stacking configuration as the starting position ($\theta=0$). In this configuration, the twisting center is placed at the center of a hexagon. For the unrotated layer, the sublattice vectors are $\tau_A = \frac{\a_1 + \a_2}{3} = d\qty(\frac{\sqrt{3}}{2}, \frac{1}{2})$ and $\tau_B = \frac{-\a_1 + 2\a_2}{3} = d(0,1)$. The sublattice vectors for the rotated layers are then given by
\begin{equation} \label{eq:tau_tbg_lattice_AB}
\tau_{\alpha}^{(\ell)} = R_{\theta_\ell} \tau_\alpha.
\end{equation}

The momentum lattice vectors for each layer are expressed as $\b_j^{(\ell)} = R_{\theta_\ell} \b_j$, with $\b_{1,2}$ defined in Equation \ref{eq:monolayer-bvecs}. The periodicity of the moiré pattern arises from the beat effect \cite{handbook2019}, leading to
\begin{equation} \label{eq:beat_effect-moire_momentum}
\b_j^{\text{moiré}} = \b_{j}^{(1)} - \b_{j}^{(2)}.
\end{equation}

Under this assumption, $K^{(1)}$ folds into $K_m'$ and $K^{(2)}$ folds into $K_m$ in the mBZ, implying that the continuum model presumes a Type I (SE-odd) structure. From Equation \ref{eq:beat_effect-moire_momentum}, the explicit expressions for the moiré momentum lattice vectors are
\begin{equation} \label{eq:moire-bvecs}
\vb{b}_1^{\text{moiré}} = \sqrt{3} k_\theta \qty(-\frac{1}{2}, -\frac{\sqrt{3}}{2}), \quad
\vb{b}_2^{\text{moiré}} = \sqrt{3} k_\theta \qty(1, 0),
\end{equation}
where $k_\theta = \abs{K^{(1)}-K^{(2)}} = 2 \abs{K} \sin(\theta/2) = \frac{8\pi}{3a} \sin(\theta/2)$.

The Hamiltonian $H$ is constructed from a tight-binding approach, $H = H_1 + H_2 + H_{\perp}$, where $H_\ell$ corresponds to the Hamiltonian of layer $\ell$ and $H_\perp = V_{12} + V_{12}^\d$ accounts for interlayer hybridization. In the Bloch wave basis, the wavefunctions are written as
\begin{equation} \label{eq:BM-blochwave}
\ket{\psi_{\ell, \k, \alpha}} = \frac{1}{\sqrt{N_\ell}} \sum_{\R_\ell} e^{i \k \vdot \qty(\R_\ell + \vtau_{\alpha}^{(\ell)})} \ket{\ell, \R_\ell, \alpha},
\end{equation}
where $\ell$ labels the layer, $N_\ell$ is the number of unit cells, $\R_\ell = n_1 \a_1^{(\ell)} + n_2 \a_2^{(\ell)}$ are the Bravais lattice positions, $\vtau_{\alpha}^{(\ell)}$ are sublattice centers, and $\ket{\ell,\R_\ell,\alpha}$ represents localized Wannier states.

%In this basis, considering only nearest-neighbor intralayer hopping, the low-energy Hamiltonian of each layer is
%\begin{equation} \label{eq:blg-eachlayer-hamil-lowenergy}
%H^{\pm \K}(\q) = \hbar v_F \abs{\q}
%\begin{pmatrix}
%0 & e^{\mp i (\theta_\q - \theta_\ell)} \\
%e^{\pm i (\theta_\q - \theta_\ell)} & 0
%\end{pmatrix},
%\end{equation}
%where we expanded $\k = \pm \K_\ell + \q$ to first-order in $\q$.

The interlayer Hamiltonian in the second quantization formalism is given by \cite{handbook2019}:
\begin{equation} \label{eq:interlayer-hopping}
V_{12} = \sum_{\R_1,\alpha,\R_2,\beta} c_{1,\alpha}^\d(\R_1) t_{12}^{\alpha\beta}(\R_1,\R_2) c_{2,\beta}(\R_2), \quad
t_{12}^{\alpha\beta}(\R_1,\R_2) =
\mel{1,\R_1,\alpha}{V_{12}}{2,\R_2,\beta}.
\end{equation}

By performing a Fourier transformation,
\begin{equation} \label{eq:blg-fourier}
c_{\ell,\alpha}^\d(\R_\ell) = \frac{1}{\sqrt{N_\ell}} \sum_{\k_\ell}
e^{-i\k_\ell \vdot \qty(\R_\ell + \vtau_{\alpha}^{(\ell)})} c_{\ell,\alpha}^\d(\k_\ell),
\end{equation}
where the summation over $\k_\ell$ spans the Brillouin zone (BZ) of layer $\ell$, the interlayer Hamiltonian becomes:
\begin{equation} \label{eq:interlayer-hopping-kspace}
V_{12} = \sum_{\k_1,\alpha,\k_2,\beta} c_{1,\alpha}^\d(\k_1) T_{12}^{\alpha\beta}(\k_1,\k_2) c_{2,\beta}(\k_2),
\end{equation}
with
\begin{equation} \label{eq:interlayer-hopping-Tkspace}
T_{12}^{\alpha\beta}(\k_1,\k_2) =
\frac{1}{\sqrt{N_1 N_2}} \sum_{\R_1,\R_2} e^{-i\k_1\vdot\qty(\R_1+\vtau_{\alpha}^{(1)})}
t_{12}^{\alpha\beta}(\R_1,\R_2) e^{i\k_2\vdot\qty(\R_2+\vtau_{\beta}^{(2)})}.
\end{equation}

Assuming that the interlayer hopping term \( t_{12}^{\alpha\beta}(\R_1,\R_2) \) depends only on the relative separation of the two orbital centers, we can apply a Fourier transformation:
\begin{equation} \label{eq:interlayer-hopping-fourier}
t_{12}^{\alpha\beta}(\R_1,\R_2) = t_{12}^{\alpha\beta}\qty(\R_1+\vtau_{\alpha}^{(1)}-\R_2-\vtau_{\beta}^{(2)}) =
\int \frac{\dd[2]{\p}}{(2\pi)^2} e^{i \p \vdot \qty(\R_1+\vtau_{\alpha}^{(1)}-\R_2-\vtau_{\beta}^{(2)})} t_{12}^{\alpha\beta}(\p).
\end{equation}

Substituting Eq.~\eqref{eq:interlayer-hopping-fourier} into Eq.~\eqref{eq:interlayer-hopping-Tkspace}, we obtain:
\begin{equation} \label{eq:interlayer-hopping-step1}
T_{12}^{\alpha\beta}(\k_1,\k_2) =
\frac{1}{\sqrt{N_1 N_2}} \int \frac{\dd[2]{\p}}{(2\pi)^2} \sum_{\R_1} e^{-i(\k_1-\p)\vdot\qty(\R_1+\vtau_{\alpha}^{(1)})}
t_{12}^{\alpha\beta}(\p) \sum_{\R_2} e^{i(\k_2-\p)\vdot\qty(\R_2+\vtau_{\beta}^{(2)})}.
\end{equation}

Using the relation
\[
\sum_{\R_\ell} e^{i \k \vdot \R_\ell} = N_\ell \sum_{\G_\ell} \delta_{\k,\G_\ell},
\]
where the summation is over momentum lattice vectors run over the layers \(\G_\ell = m_1 \b_1^{(\ell)} + m_2 \b_2^{(\ell)}\), Eq.~\eqref{eq:interlayer-hopping-step1} can be rewritten as:
\begin{equation} \label{eq:interlayer-hopping-step2}
T_{12}^{\alpha\beta}(\k_1,\k_2) =
\sqrt{N_1 N_2} \int \frac{\dd[2]{\p}}{(2\pi)^2} \sum_{\G_1, \G_2}
e^{-i \G_1 \vdot \vtau_\alpha^{(1)}} t_{12}^{\alpha\beta}(\p) e^{i \G_2\vdot\vtau_\beta^{(2)}}
\delta_{\k_1-\p, \G_1} \delta_{\k_2-\p, \G_2}.
\end{equation}

Using the Dirac delta property \(\int \dd[2]{\k} \delta_{\k-\k'} = \frac{(2\pi)^2}{A}\), where \(A\) is the total area of the system, we can write:
\begin{equation} \label{eq:interlayer-hopping-step3}
T_{12}^{\alpha\beta}(\k_1,\k_2) =
\sqrt{\frac{N_1 N_2}{A^2}} \sum_{\G_1, \G_2}
e^{i \G_1 \vdot \vtau_\alpha^{(1)}} t_{12}^{\alpha\beta}(\k_1+\G_1) e^{-i \G_2\vdot\vtau_\beta^{(2)}}
\delta_{\k_1+\G_1, \k_2+\G_2}.
\end{equation}

By expressing \(A\) as \(A = N_\ell A_1\), where \(A_1 = \frac{\sqrt{3}}{2} a^2\) represents the unit cell area of each layer, the above simplifies to:
\begin{equation} \label{eq:interlayer-hopping-simplified}
T_{12}^{\alpha\beta}(\k_1,\k_2) = \frac{1}{A_1} \sum_{\G_1, \G_2}
e^{i \G_1 \vdot \vtau_\alpha^{(1)}} t_{12}^{\alpha\beta}(\k_1+\G_1) e^{-i \G_2\vdot\vtau_\beta^{(2)}}
\delta_{\k_1+\G_1, \k_2+\G_2}.
\end{equation}
Here, the condition \(\k_1 + \G_1 = \k_2 + \G_2\) is known as the generalized umklapp condition \cite{handbook2019}.

\textbf{TEXTO DO ANGELI ++++++++++++++++++++++++++++}

Since we are interested in the low-energy physics, \( \mathbf{k} \) and \( \mathbf{p} \) must lie close to their respective Dirac points: \( \mathbf{K}_\phi \) and \( \mathbf{K}'_\phi \) for \( \mathbf{p} \) in layer 1, and \( \mathbf{K}_{-\phi} \) and \( \mathbf{K}'_{-\phi} \) for \( \mathbf{k} \) in layer 2. Consequently, \( \hat{T}_{\mathbf{k}\mathbf{p}} \) can, in principle, couple states within the same valley of different layers or states between opposite valleys. However, since \( T_\perp(\mathbf{q}) \) decays exponentially with \( \mathbf{q} = \abs{\mathbf{q}} \) \cite{tperp-laissardiere2012}, the leading contributions arise from terms with the smallest possible \( \abs{\mathbf{k} + \mathbf{G}^{(2)}} \) that satisfy momentum conservation:
\[
\mathbf{k} + \mathbf{G}^{(2)} = \mathbf{p} + \mathbf{G}^{(1)}.
\]

For small twist angles \( \phi \), only the intra-valley matrix elements (\( \mathbf{p} \sim \mathbf{k} \)) are significant, while inter-valley contributions are negligibly small. This remains true despite opposite valleys of different layers folding into the same point in the moiré Brillouin zone (MBZ). For example, if \( \mathbf{p} \approx \mathbf{K}_\phi \) and \( \mathbf{k} \approx \mathbf{K}'_{-\phi} \), momentum conservation requires very large reciprocal lattice vectors, specifically \( \mathbf{G}^{(1)} = (2k+1)(\mathbf{G}^{(1)}_a - \mathbf{G}^{(1)}_b) \) and \( \mathbf{G}^{(2)} = (2k+1)(\mathbf{G}^{(2)}_b - \mathbf{G}^{(2)}_a) \), resulting in an exponentially small \( T_\perp(\abs{\mathbf{k} + \mathbf{G}^{(2)}}) \).

This effective decoupling of the two valleys ensures that the number of electrons within each valley is, to a high degree of accuracy, a conserved quantity. Consequently, there emerges an approximate valley \( U_v(1) \) symmetry \cite{tperp-laissardiere2012, 69}, which leads to accidental band degeneracies along high-symmetry directions in the MBZ.

\textbf{FINAL DO TEXTO DO ANGELI ++++++++++++++++++++++}

Equation \eqref{eq:interlayer-hopping-simplified} relies on the functional form of \( t_{12}^{\alpha\beta}(\mathbf{p}) \). To make further progress, we introduce additional assumptions. Since both \( A \) and \( B \) sites in graphene correspond to \( p_z \) orbitals of carbon atoms, we assume that \( t_{12}^{\alpha\beta}(\mathbf{r}) = t_\perp(\mathbf{r}) \) is independent of \( \alpha \) and \( \beta \). In \cite{tperp-laissardiere2012}, the authors characterize the behavior of \( t_\perp(\mathbf{r}) \) using Slater-Koster parameters and numerically evaluate \( t_\perp(\mathbf{p}) \) through Eq. \eqref{eq:interlayer-hopping-fourier}. They establish that \( t_\perp(\mathbf{p}) \) depends solely on \( \abs{\mathbf{p}} \) and decays exponentially. This observation is significant, as it ensures that only a few umklapp processes contribute to the coupling in Eq. \eqref{eq:interlayer-hopping-simplified}.

In the small-angle limit \( \theta \lesssim 10^\circ \), we expand \( \mathbf{k}_\ell = K^{(\ell)}+ \q_\ell \) around the Dirac points of each layer, with \( \abs{\q_\ell} \sim k_\theta \ll \abs{K} \). Approximating \( t_\perp(K^{(1)}+ \q_1 + \G_1) \approx t_\perp(K^{(1)} + \G_1) \), the interlayer coupling becomes
\begin{equation} \label{eq:interlayer-hopping-truncation}
T_{12}^{\alpha\beta}(\q_1, \q_2) = \frac{1}{A_{1}} \sum_{\G_1, \G_2} e^{i \G_1 \cdot \bm{\tau}_{\alpha}^{(1)}}
t_{\perp}(K_1 + \G_1) e^{-i \G_2 \cdot \bm{\tau}_{\beta}^{(2)}}
\delta_{\K_1 + \q_1 + \G_1, \K_2 + \q_2 + \G_2}.
\end{equation}

Since \( t_\perp(\mathbf{p}) \) decays exponentially with \( \abs{\mathbf{p}} \), we truncate \( t_\perp(\mathbf{p}) \approx 0 \) for \( \abs{\mathbf{p}} > \abs{K} \). Thus, the quantity $t_\perp(K_1 + \G_1)$ is only significant for values of $K_1 + \G_1$ inside the BZ of the first layer. Since $\G_1$ is a lattice vector of the first layer, this will only happen if $K_1 + \G_1$ is equivalent to some of the equivalent Dirac points $K_1$, $C_{3} K_1$ or $C_{3}^2 K_1$. This leaves only three possible values for $\G_1 = \0, -\b_1^{(1)}, -(\b_1^{(1)}+\b_2^{(1)})$. Because of the term $\delta_{\K_1 + \q_1 + \G_1, \K_2 + \q_2 + \G_2}$, something analogous must happen with $\G_2 = \0, -\b_1^{(2)}, -(\b_1^{(2)}+\b_2^{(2)})$. In summary, there are three possibilities:
\begin{align} \label{eq:G1G2_3_possibilities}
\begin{cases}
\; \G_1^{(\text{b})} = \0,                      & \G_2^{(\text{b})} = \0; \\
\; \G_1^{(\text{tr})} = -\b_1^{(1)},            & \G_2^{(\text{tr})} = \b_1^{(2)}; \\
\; \G_1^{(\text{tl})} = -\b_1^{(1)}-\b_2^{(1)}, & \G_2^{(\text{tl})} = -\b_1^{(2)}-\b_2^{(2)}.
\end{cases}
\end{align}

Also, rewriting the Delta term as
\begin{equation} \label{eq:delta-term}
\delta_{\K_1 + \q_1 + \G_1, \K_2 + \q_2 + \G_2} = \delta_{\q_2 - \q_1, \K_1 - \K_2 + \G_1 - \G_2},
\end{equation}
we conclude that the three possibilities in Equation \ref{eq:G1G2_3_possibilities} result in three possible momentum transfers bottom (b), top right (tr), and top left (tl) for $\q_2-\q_1$:
\begin{equation} \label{eq:qb_qtr_qtl}
\begin{cases}
\; \q_\text{b} = K_1 - K_2 = k_\theta \, (0, -1), \\
\; \q_\text{tr} = K_1 - K_2 - \b_1^{\text{moiré}} = k_\theta \, \qty(\frac{\sqrt{3}}{2}, \frac{1}{2}), \\
\; \q_\text{tl} = K_1 - K_2 -\b_1^{\text{moiré}}-\b_2^{\text{moiré}} = k_\theta \, \qty(-\frac{\sqrt{3}}{2}, \frac{1}{2}).
\end{cases}
\end{equation}
%\begin{equation} \label{eq:qb}
%\q_\text{b} = \K_1 - \K_2 = k_\theta \, (0, -1),
%\end{equation}
%\begin{equation} \label{eq:qtr}
%\q_\text{tr} = (\K_1 - \K_2) + (\g_{12} - \g_{22}) = k_\theta \, \qty(\frac{\sqrt{3}}{2}, \frac{1}{2}),
%\end{equation}
%\begin{equation} \label{eq:qtl}
%\q_\text{tl} = (\K_1 - \K_2) + (\g_{13} - \g_{23}) = k_\theta \, \qty(-\frac{\sqrt{3}}{2}, \frac{1}{2}).
%\end{equation}

Ultimately, the interlayer coupling reduces to three terms:
\begin{equation} \label{eq:interlayer-3terms}
T_{12}^{\alpha\beta}(\q_1, \q_2) = T^{\alpha\beta}_{\q_\text{b}} \delta_{\q_2 - \q_1, \q_\text{b}}
+ T^{\alpha\beta}_{\q_\text{tr}} \delta_{\q_2 - \q_1, \q_\text{tr}}
+ T^{\alpha\beta}_{\q_\text{tl}} \delta_{\q_2 - \q_1, \q_\text{tl}},
\end{equation}
where
\begin{equation} \label{eq:interlayer-tensor}
T^{\alpha\beta}_{\q_{\text{b},\text{tr},\text{tl}}} = \frac{t_\perp(\abs{\K})}{A_{\text{u.c.}}} e^{i \G_{1}^{(\text{b},\text{tr},\text{tl})} \cdot \bm{\tau}_{\alpha}^{(1)}}
e^{-i \G_{2}^{(\text{b},\text{tr},\text{tl})} \cdot \bm{\tau}_{\beta}^{(2)}}.
\end{equation}

In the \( A, B \) basis, the matrices for each term are:
\begin{equation} \label{eq:T-qb}
T_{\q_\text{b}} = \frac{t_\perp(\abs{\K})}{A_{\text{u.c.}}}
\begin{pmatrix}
1 & 1 \\
1 & 1
\end{pmatrix},
\end{equation}
\begin{equation} \label{eq:T-qtr}
T_{\q_\text{tr}} = \frac{t_\perp(\abs{\K})}{A_{\text{u.c.}}}
\begin{pmatrix}
1 & e^{2\pi i/3} \\
e^{-2\pi i/3} & 1
\end{pmatrix},
\end{equation}
\begin{equation} \label{eq:T-qtl}
T_{\q_\text{tl}} = \frac{t_\perp(\abs{\K})}{A_{\text{u.c.}}}
\begin{pmatrix}
1 & e^{-2\pi i/3} \\
e^{2\pi i/3} & 1
\end{pmatrix}.
\end{equation}

For the valley $K'$ the derivation is analogous, which leads us to
\begin{equation} \label{eq:interlayer-3terms_valley-1}
T^{\zeta=-1}(\q_1, \q_2) = T_{\q_\text{b}}^* \delta_{\q_2 - \q_1, \q_\text{b}}
+ T_{\q_\text{tr}}^* \delta_{\q_2 - \q_1, \q_\text{tr}}
+ T_{\q_\text{tl}}^* \delta_{\q_2 - \q_1, \q_\text{tl}},
\end{equation}

\textbf{TEXTO DO ANGELI ++++++++++++++++++++++++++++}

Let us briefly discuss how lattice relaxation effects can be incorporated into the model. As highlighted in the previous section, lattice relaxation tends to compress the energetically unfavorable AA regions while enlarging the Bernal-stacked triangular domains in the moiré pattern. This modification influences the amplitudes of both inter-sublattice and intra-sublattice hopping processes. These effects are captured by adjusting the operators \( \hat{T}_i \) in Eq. (2.27) as follows:
\[
\hat{T}_1 \to T_1(u, u') =
\begin{pmatrix}
u & u' \\
u' & u
\end{pmatrix}
= u \sigma_0 + u' \sigma_x,
\]
\[
\hat{T}_2 \to \hat{T}_2(u, u') =
\begin{pmatrix}
u & u' \omega^* \\
u' \omega & u
\end{pmatrix}
= u \sigma_0 + u'
\left(
\cos \frac{2\pi}{3} \sigma_x + \sin \frac{2\pi}{3} \sigma_y
\right),
\]
\[
\hat{T}_3 \to \hat{T}_3(u, u') =
\begin{pmatrix}
u & u' \omega \\
u' \omega^* & u
\end{pmatrix}
= u \sigma_0 + u'
\left(
\cos \frac{2\pi}{3} \sigma_x - \sin \frac{2\pi}{3} \sigma_y
\right),
\]
where \( u \) is generally smaller than \( u' \), reflecting the asymmetry introduced by lattice relaxation.

In **Fig. 2.6**, we show the band structure of twisted bilayer graphene at \( \theta = 1.08^\circ \), obtained using the \( u \) and \( u' \) parameters reported in Ref. [53]. The resulting band structure closely resembles that obtained via tight-binding models, with the notable feature that the bands exhibit particle-hole symmetry.

\textbf{FINAL DO TEXTO DO ANGELI ++++++++++++++++++++++++++++}

To obtain the final hamiltonian for the BM model, it is useful to represent it in a new lattice \cite{all_magic_angles}. We define the set of vectors
\begin{equation} \label{eq:Q_lattice_QA_QB_def}
\{\Q_A, \Q_B\} =
\begin{cases}
\; \Q_A = \K_2 + m \b_1^{\text{moiré}} + n \b_2^{\text{moiré}}; \\
\; \Q_B = \K_1 + m \b_1^{\text{moiré}} + n \b_2^{\text{moiré}},
\end{cases}
\end{equation}
where $\Q_A$ represent the black lattice and $\Q_B$ the red lattice as in \cite{thesis_angeli, all_magic_angles}. They span the vertices of the mBZs, where $\Q_A$ (black circles) correspond to valley $\zeta = -1$ in layer 1 and valley $\zeta = +1$ in layer 2, while $\Q_B$ (red circles) correspond to valley $\zeta = +1$ in layer 1 and valley $\zeta = -1$ in layer 2.

\begin{figure}[H]
\centering
\includegraphics[width=0.8\linewidth]{fig/moire-vectors.png}
\caption{\textbf{FAZER MINHA PRÓPRIA FIGURA.}}
\end{figure}

Next, we redefine the momenta for layers 1 and 2 as
\begin{equation} \label{eq:redef_k1k2_momenta_Qlattice}
\q_1 = \k_1 - K_1 = \k - \Q_B, \quad \q_2 = \k_2 - K_2 = \k - \Q_A.
\end{equation}

Since $\q_2 - \q_1$ can assume the three values $\q_{\text{b}}$, $\q_{\text{tr}}$ or $\q_{\text{tl}}$, in the new representation the selection rules becomes
\begin{align} \label{eq:selection_rules_qbqtrqtl_QAQB}
\begin{cases}
\; \q_2-\q_1=\q_{\text{b}}  \implies \Q_B = \Q_A + \q_{\textbf{b}} \\
\; \q_2-\q_1=\q_{\text{tr}} \implies \Q_B = \Q_A + \q_{\textbf{tr}} \\
\; \q_2-\q_1=\q_{\text{tl}} \implies \Q_B = \Q_A + \q_{\textbf{tl}}
\end{cases}
\end{align}

With those definitions, the Hamiltonian of valley $\zeta$ reads
\begin{equation} \label{eq:mbm-1v}
H^\zeta_{\Q\Q'}(\k) = \zeta \delta_{\Q,\Q'} v_F (\k - \Q) \vdot (\s_x, \zeta \s_y) +
\sum_{j = \text{b},\text{tr},\text{tl}} \qty(\delta_{\Q'-\Q,\q_j} + \delta_{\Q-\Q',\q_j}) T_{j}^\zeta.
\end{equation}

<++> <++> <++> <++> <++> <++> <++> <++>

%%%%%%%%%%%%%%%%%%%%%%%%%%%%%%%%%%%%%%%%%%%%%%%%%%%%%%%%%%%%%%%%%%%%%%%%%%%%%%%%%%%%%%%%%%%%%%%%%%
\section{Symmetry analysis and Wannier obstruction} \label{sec:wannier_obstruction}
%%%%%%%%%%%%%%%%%%%%%%%%%%%%%%%%%%%%%%%%%%%%%%%%%%%%%%%%%%%%%%%%%%%%%%%%%%%%%%%%%%%%%%%%%%%%%%%%%%

As discussed in \cite{zou2018}, at currently accessible energy scales, experiments indicate that MATBG exhibits some approximate symmetries. These include translational symmetry characterized by the constant $L' = a / (2 \sin\theta/2)$, valley symmetry $U_v(1)$, $C_{2z} \mathcal{T}$, and $D_6$ point group symmetry, which are not fully captured by a generic commensurate structure described in Section \ref{sec:tbg_geom}. The continuum model outlined in Section \ref{sec:BM-model} incorporates all these ``good'' symmetries. In this section, we will explore group theory concepts that will later be applied to understand the Wannier Obstruction \cite{zou2018} and the Topological Heavy Fermion model of MATBG \cite{topoheavyfermion2022}.

For our purposes, the MATBG system exhibits emergent symmetries compatible with the $P622$ space group \cite{thesis_rennella}, which is symmorphic and associated with the $D_6$ point group.

For TBG, we focus on the Wyckoff position $2c$ (\textbf{NO! WE DO NOT!}), corresponding to the AB and BA regions of interest in Figure \ref{fig:moireD6}. The site-symmetry groups of these regions are all isomorphic to the point group $D_3$, a subgroup of $D_6$. In contrast, the $AA$ regions belong to Wyckoff position $1a$, and their site-symmetry groups have higher symmetry $D_6$.

The BM model is advantageous as it is independent of whether a configuration of TBG is commensurate or not, and it also incorporates all the emergent symmetries observed in experiments \cite{zou2018}.

%%%%%%%%%%%%%%%%%%%%%%%%%%%%%%%%%%%%%%%%%%%%%%%%%%%%%%%%%%%%%%%%%%%%%%%%%%%%%%%%%%%%%%%%%%%%%%%%%%
\section{Monolayer}
%%%%%%%%%%%%%%%%%%%%%%%%%%%%%%%%%%%%%%%%%%%%%%%%%%%%%%%%%%%%%%%%%%%%%%%%%%%%%%%%%%%%%%%%%%%%%%%%%%

Considering only nearest-neighbors, the hamiltonian of a single unrotated layer of graphene is given by
$$
H_{\text{mono}}(\k) = -t
\begin{pmatrix}
0 & f(\k) \\
f^*(\k) & 0
\end{pmatrix}
, \quad \ell = 1, 2,
$$
where $f = \sum_{i=1}^{3} e^{i \k \vdot \bm{\delta}}$, and $\bm{\delta}$ are the three nearest-neighbor vectors from a site $A$ of the honeycomb lattice.

Expanding in first-order $\k = \K + \q$, where $\K = \frac{4\pi}{3a} (1, 0)$ is the Dirac point of the unrotated layer, we have
$$
H_{\text{mono}}(\K + \q) \approx v_F \, \q \vdot \bm{\sigma}.
$$

%%%%%%%%%%%%%%%%%%%%%%%%%%%%%%%%%%%%%%%%%%%%%%%%%%%%%%%%%%%%%%%%%%%%%%%%%%%%%%%%%%%%%%%%%%%%%%%%%%
\section{BM model review}
%%%%%%%%%%%%%%%%%%%%%%%%%%%%%%%%%%%%%%%%%%%%%%%%%%%%%%%%%%%%%%%%%%%%%%%%%%%%%%%%%%%%%%%%%%%%%%%%%%

The hamiltonian of the BM model is given by
$$
H = H_1 + H_2 + V + V^\dagger,
$$
where $H_1$ and $H_2$ correspond to the layers 1 and 2, and $V$ is the hybridization between them.

\n

For this discussion, we use the reference frame where layer 1 is unrotated and layer 2 is rotated counter-clockwise by an angle $\theta$. Therefore, we have
$$
H_1(\k) = H_{\text{mono}}(\k), \quad H_2(\k) = H_1(R_\theta^{-1}\k),
$$
$$
R_\theta =
\begin{pmatrix}
\cos\theta & -\sin\theta \\
\sin\theta & \cos\theta
\end{pmatrix}.
$$

Also, the Dirac points of each layer are
$$
\K_1 = \K = \frac{4\pi}{3a} (1, 0), \quad \K_2 = R_\theta \K_1 = R_\theta \K,
$$
$$
$$

Therefore, expanding around the Dirac points $\K_1$ and $\K_2$, respectively:
$$
H_1(\K_1 + \q) \approx v_F \, \q \vdot \bm{\sigma}.
$$
$$
\boxed{ H_2(\K_2 + \q) = H_2(R_\theta \K_1 + \q) = H_1(\K_1 + R_\theta^{-1}\q) \approx v_F \, (R_\theta^{-1}\q) \vdot \bm{\sigma} \approx v_F \, (1 - i \theta \sigma_y ) \, \q \vdot \bm{\sigma}. }
$$

If we \textbf{neglect the $\theta$-dependence on $H_2$}, our BM model will be \textbf{particle-hole symmetric}, as defined by Bernevig.

\n

Of course, the more difficult part is due to the hybridization term $V$. It is written on the Bloch basis $\ket{\ell, \alpha, \p}$, where $\ell = 1, 2$ is the layer index, $\alpha = A, B$ is the sublattice index, and $\p$ is the momentum.
We have
$$
V_{\alpha\beta}(\p, \p') = \bra{1, \alpha, \p} H \ket{2, \beta, \p'}.
$$

When we expand both $\p = \K_1 + \q$ and $\p' = \K_2 + \q'$, \textbf{after the BM considerations}, we get
$$
V_{\alpha\beta}(\K + \q, R_\theta \K + \q') \approx
w \sum_{j=1}^{3} \delta_{\q, \q' + \q_j} T_{\alpha\beta}^j,
$$
where $\q_1, \q_2, \q_3$ are the moiré momentum vectors, with absolute value $k_D = 2 \sin(\theta/2) \abs{\K}$. The matrices $T^j_{\alpha\beta}$ are
$$
T_1 = \sigma_0 + \sigma_x
$$
$$
T_2 = \sigma_0 + \cos(\frac{2\pi}{3}) \sigma_x + \sin(\frac{2\pi}{3}) \sigma_y
$$
$$
T_3 = \sigma_0 + \cos(\frac{2\pi}{3}) \sigma_x - \sin(\frac{2\pi}{3}) \sigma_y
$$
\begin{figure}[H]
\centering
\includegraphics[width=0.8\linewidth]{fig/moire-vectors.png}
\end{figure}

\n

This \textbf{BM model} is called \textbf{MBM-1V (Moiré Band Model - One Valley)} by Bernevig. If we decompose $\q = \k - \Q$, $\q'=\k-\Q'$, where $\Q$ and $\Q'$ belong to the hexagonal lattice formed by adding $\q_{1,2,3}$ iteratively, we can rewrite the hamiltonian as
$$
\boxed{
H_{\Q, \Q'}^{(\text{MBM-1V})}(\k) =
\delta_{\Q,\Q'} v_F (\k-\Q) \vdot \bm{\sigma}
+ w \sum_{j=1}^{3} (\delta_{\Q'-\Q, \q_j} + \delta_{\Q-\Q', \q_j}) T^j.
}
$$

\begin{itemize}
\item This model is \textbf{topological} and is what we have been discussing the whole time.
\item The tables of Bernevig apply to this model, where we consider the Wyckoff position 1a within the THF approach to solve the topological obstruction.
\item Its magnetic space group is $P6'2'2$. Generators: $C_{6z} T$, $C_{2y} T$, $C_{2x}$.
\end{itemize}

\begin{table}[H]
\scriptsize
\caption{Elementary band representations of the magnetic space group $P6'2'2$.}
\centering
\begin{tabular}{|c|c|c|c|c|c|c|c|c|}
\hline
Wyckoff & \multicolumn{3}{c|}{$1a$} & \multicolumn{3}{c|}{$2c$} & \multicolumn{2}{c|}{$3f$} \\
\cline{1-9}
Site sym. & \multicolumn{3}{c|}{$6'2'2$, $32$} & \multicolumn{3}{c|}{$32$, $32$} & \multicolumn{2}{c|}{$2'2'2$, $2$} \\
\cline{1-9}
EBR & $G_{A_1}^{1a}(1)$ & $G_{A_2}^{1a}(1)$ & $G_{E}^{1a}(2)$ & $G_{A_1}^{2c}(2)$ & $G_{A_2}^{2c}(2)$ & $G_{E}^{2c}(4)$   & $G_{A}^{3f}(3)$ & $G_{B}^{3f}(3)$ \\
\hline
$\Gamma$ & $\Gamma_1(1)$ & $\Gamma_2(1)$ & $\Gamma_3(1)$ & $2\Gamma_1(1)$ & $2\Gamma_2(1)$ & $2\Gamma_3(2)$ & $\Gamma_1(1)+\Gamma_3(2)$ & $\Gamma_2(1)+\Gamma_3(2)$ \\
\hline
$K$ & $K_1(1)$ & $K_1(1)$ & $K_2 K_3(2)$ & $K_2 K_3(2)$ & $K_2 K_3(2)$ & $2K_1(1) + K_2 K_3(2)$ & $K_1(1)+K_2 K_3(2)$ & $K_1(1)+K_2 K_3(2)$ \\
\hline
$M$ & $M_1(1)$ & $M_2(1)$ & $M_1(1)+M_2(1)$ & $2M_1(1)$ & $2M_2(1)$ & $2M_1(1)+2M_2(1)$ & $2M_1(1)+M_2(1)$ & $M_1(1)+2M_2(2)$ \\
\hline
\end{tabular}
\label{tab:ebr-P6'2'2}
\end{table}

\begin{table}[H]
\caption{Character table of irreps at high symmetry momenta in magnetic space group $P6'2'2$.}
\centering
\begin{tabular} { c c c c | c c c | c c c }
\cline{1-10}
$\P$ & $\P \Gamma_1$ & $\P \Gamma_2$ & $\P \Gamma_3$ & $\P$ & $\P M_1$ & $\P M_2$ & $\P$ & $\P K_1$ & $\P K_2K_3$ \\
\cline{1-10}
$E$ & $\P1$ & $\P1$ & $\P2$ & $\P E$ & $\P1$ & $\P1$ & $\P E$ & $\P1$ & $\P2$ \\
$2 C_3$ & $\P1$ & $\P1$ & $ -1$ & $\P C_2'$ & $\P1$ & $ -1$ & $\P C_3$ & $\P1$ & $ -1$ \\
$3 C_2'$ & $\P1$ & $ -1$ & $\P0$ & $\P$ & $\P$ & $\P$ & $\P C_3^{-1}$ & $\P1$ & $-1$ \\
\cline{1-10}
\end{tabular}
\label{tab:char-P6'2'2}
\end{table}

%%%%%%%%%%%%%%%%%%%%%%%%%%%%%%%%%%%%%%%%%%%%%%%%%%%%%%%%%%%%%%%%%%%%%%%%%%%%%%%%%%%%%%%%%%%%%%%%%%
\section{There exists another model}
%%%%%%%%%%%%%%%%%%%%%%%%%%%%%%%%%%%%%%%%%%%%%%%%%%%%%%%%%%%%%%%%%%%%%%%%%%%%%%%%%%%%%%%%%%%%%%%%%%

According to Bernevig: ``The MBM-1V is half of the TBG system, with similar physics taking place in the electron states around $\K'$''. A model with the two valleys can be written as
$$
\boxed{
H_{\Q, \Q'}^{(\text{MBM-2V})}(\k) =
\delta_{\Q,\Q'} v_F (\k-\Q) \vdot \bm{\sigma} \otimes \tau_z
+ w \sum_{j=1}^{3} (\delta_{\Q'-\Q, \q_j} + \delta_{\Q-\Q', \q_j}) T^j \otimes \tau_0.
}
$$

Here $\tau_z$ and $\tau_0$ are the Pauli and identity matrix representing the valley degree of freedom.

\begin{itemize}
\item This model is \textbf{apparently different} from the latter, and \textbf{is not topological}. You can decompose it in EBR's as $G_{A_1}^{2c} + G_{A_2}^{2c}$.
\item The tables of Rennella apply to this model. There the authors (Vafek, Rennella, Angeli, Koshino, etc) consider Wyckoff position 2c, because these positions really correspond to symmetry-adapted exponentially localized Wannier functions.
\item The magnetic space group is $P6221'$. Generators: $C_{6z}$, $C_{2y}$, $C_{2x}$, $T$.
\end{itemize}

\begin{table}[H]
\caption{Elementary band representations generated from Wyckoff position $2c$ of the space group $P622$.}
\centering
\begin{tabular}{|c|c|c|c|}
\hline
Wyckoff & \multicolumn{3}{c|}{$2c$} \\
\cline{1-4}
Site sym. & \multicolumn{3}{c|}{$32$} \\
\cline{1-4}
EBR & $G_{A_1}^{2c}(2)$ & $G_{A_2}^{2c}(2)$ & $G_{E}^{2c}(4)$  \\
\hline
$\Gamma$ & $\Gamma_1(1) + \Gamma_4(1)$ & $\Gamma_2(1) + \Gamma_3(1)$ & $\Gamma_5(2) + \Gamma_6(2)$ \\
\hline
$K$ & $K_3(2)$ & $K_3(2)$ & $K_1(1) + K_2(1) + K_3(2)$ \\
\hline
$M$ & $M_1(1) + M_4(1)$ & $M_2(1) + M_3(1)$ & $M_1(1) + M_2(1) + M_3(1) + M_4(1)$ \\
\hline
\end{tabular}
\label{tab:ebr-P622}
\end{table}

\begin{table}[H]
\caption{Character table of irreps at high symmetry momenta in space group $P622$.}
\scriptsize
\centering
\begin{tabular} { c c c c c c c | c c c c c | c c c c }
\cline{1-16}
$\P$ & $\P \Gamma_1$ & $\P \Gamma_2$ & $\P \Gamma_3$ & $\P \Gamma_4$ & $\P \Gamma_5$ & $\P \Gamma_6$ & $\P$ & $\P M_1$ & $\P M_2$ & $\P M_3$ & $\P M_4$ & $\P$ & $\P K_1$ & $\P K_2$ & $\P K_3$\\
\cline{1-16}
$E$      & $\P1$ & $\P1$ & $\P1$ & $\P1$ & $\P2$ & $\P2$ & $E$     & $\P1$ & $\P1$  & $\P1$ & $\P1$ & $E$      & $\P1$ & $\P1$ & $\P2$ \\
$2C_6$   & $\P1$ & $\P1$ & $ -1$ & $ -1$ & $ -1$ & $\P1$ & $C_2$   & $\P1$ & $\P1$  & $ -1$ & $ -1$ & $C_3$    & $\P1$ & $\P1$ & $ -1$ \\
$2C_3$   & $\P1$ & $\P1$ & $\P1$ & $\P1$ & $ -1$ & $ -1$ & $C_2'$  & $\P1$ & $ -1$  & $ -1$ & $\P1$ & $3C_2''$ & $\P1$ & $ -1$ & $\P0$ \\
$C_2$    & $\P1$ & $\P1$ & $ -1$ & $ -1$ & $\P2$ & $ -2$ & $C_2''$ & $\P1$ & $ -1$  & $\P1$ & $ -1$ &          &       &       &       \\
$3C_2'$  & $\P1$ & $ -1$ & $ -1$ & $\P1$ & $\P0$ & $\P0$ &         &       &        &       &       &          &       &       &       \\
$3C_2''$ & $\P1$ & $ -1$ & $\P1$ & $ -1$ & $\P0$ & $\P0$ &         &       &        &       &       &          &       &       &       \\
\cline{1-16}
\end{tabular}
\label{tab:char-P622}
\end{table}

Compare tables \ref{tab:ebr-P622} and \ref{tab:char-P622} with tables 2.5 and (2.1, 2.3) of Rennella \cite{thesis_rennella}, respectively.


%%%%%%%%%%%%%%%%%%%%%%%%%%%%%%%%%%%%%%%%%%%%%%%%%%%%%%%%%%%%%%%%%%%%%%%%%%%%%%%%%%%%%%%%%%%%%%%%%%
\section{\textcolor{red}{DOWN BELOW INCOMPLETE OLD TEXT}}
%%%%%%%%%%%%%%%%%%%%%%%%%%%%%%%%%%%%%%%%%%%%%%%%%%%%%%%%%%%%%%%%%%%%%%%%%%%%%%%%%%%%%%%%%%%%%%%%%%

%%%%%%%%%%%%%%%%%%%%%%%%%%%%%%%%%%%%%%%%%%%%%%%%%%%%%%%%%%%%%%%%%%%%%%%%%%%%%%%%%%%%%%%%%%%%%%%%%%
\subsection{Bistritzer-MacDonald Model}
%%%%%%%%%%%%%%%%%%%%%%%%%%%%%%%%%%%%%%%%%%%%%%%%%%%%%%%%%%%%%%%%%%%%%%%%%%%%%%%%%%%%%%%%%%%%%%%%%%

%%%%%%%%%%%%%%%%%%%%%%%%%%%%%%%%%%%%%%%%%%%%%%%%%%%%%%%%%%%%%%%%%%%%%%%%%%%%%%%%%%%%%%%%%%%%%%%%%%
\subsection{Geometry}
%%%%%%%%%%%%%%%%%%%%%%%%%%%%%%%%%%%%%%%%%%%%%%%%%%%%%%%%%%%%%%%%%%%%%%%%%%%%%%%%%%%%%%%%%%%%%%%%%%

The atoms positions of the each layer are given by
\begin{equation} \label{eq:position-atoms-tbg}
\r = n_1 \a_{\ell,1} + n_2 \a_{\ell,2} + \vtau_{\ell,\alpha},
\end{equation}
where $\alpha = A,B$ indexes the type of the site, and $\tau_{\ell,\alpha}$ are basis vectors. The lattice vectors of each layer are the rotated $\a_{\ell,j} = R_{\theta_\ell} \a_{j}$, where
\begin{equation} \label{eq:rotation-matrix}
R_\theta =
\begin{pmatrix}
\cos\theta & -\sin\theta \\
\sin\theta & \cos\theta
\end{pmatrix},
\end{equation}
is the rotation matrix by an angle $\theta$ and we use the reference frame where $\theta_1 = -\theta/2$ and $\theta_2 = +\theta/2$.

We shall consider AB-stacking. In this case, the translation vectors are given by
\begin{align}
\tau_{1,A} &= R_{-\theta/2}(0,0), \quad \tau_{2,A} = R_{\theta/2} [(0,-d) + \vtau_0], \label{eq:tauA} \\
\tau_{1,B} &= R_{-\theta/2}(0,d), \quad \tau_{2,B} = R_{\theta/2} [(0,0) + \vtau_0], \label{eq:tauB}
\end{align}
where $\vtau_0$ is an additional arbitrary translation considered in layer 2 for the sake of generalization. For example, $\vtau_0 = (0,0)$ corresponds to AB-stacking and $\vtau_0 = (0, d)$ corresponds to AA-stacking.

%%%%%%%%%%%%%%%%%%%%%%%%%%%%%%%%%%%%%%%%%%%%%%%%%%%%%%%%%%%%%%%%%%%%%%%%%%%%%%%%%%%%%%%%%%%%%%%%%%
\subsection{Moiré pattern}
%%%%%%%%%%%%%%%%%%%%%%%%%%%%%%%%%%%%%%%%%%%%%%%%%%%%%%%%%%%%%%%%%%%%%%%%%%%%%%%%%%%%%%%%%%%%%%%%%%

Here we will review the Bistritzer-MacDonald (BM) model \cite{macdonald2011}. First of all, the moiré pattern can be interpreted by a beat effect. Define the functions $h_\ell(\r)$, for each layer $\ell = 1, 2$, that describe their periodicity
\begin{equation} \label{eq:beat-functions}
h_\ell(\r) = \sum_{k=1}^{3} \cos(\vb{G}_{\ell,k} \vdot \r),
\end{equation}
where the wave vectors $\vb{G}_{\ell,1} = \vb{b}_{\ell,1}$, $\vb{G}_{\ell,2} = \vb{b}_{\ell,2}$ and $\vb{G}_{\ell,3} = \vb{b}_{\ell,1} - \vb{b}_{\ell,2}$ give the directions for nearest neightbor hoppings in the hexagonal lattice. The moiré pattern will then be an interference between $h_1(\r)$ and $h_2(\r)$,
\begin{equation} \label{eq:beat-effect}
h_{\text{m}}(\r) = h_1(\r) + h_2(\r) =
\sum_{k=1}^{3}
\, 2 \cos(\frac{\vb{G}_{1,k}+\vb{G}_{2,k}}{2} \vdot \r) \cos(\frac{\vb{G}_{1,k}-\vb{G}_{2,k}}{2} \vdot \r).
\end{equation}

The moiré pattern oscillates with $\vb{b}_k^{\text{m}} = \vb{G}_{1,k} - \vb{G}_{2,k}$, \cite{handbook2019}.

\n

Working on a coordinate system where the layer 2 is rotated by $\theta/2$ and layer 1 by $-\theta/2$, we have
\begin{equation} \label{eq:moire-bvecs}
\vb{b}_1^{\text{m}} = \sqrt{3} \abs{\Delta \K} \qty( \frac{1}{2}, -\frac{\sqrt{3}}{2}), \quad
\vb{b}_2^{\text{m}} = \sqrt{3} \abs{\Delta \K} \qty( \frac{1}{2},  \frac{\sqrt{3}}{2}),
\end{equation}
where $\abs{\Delta \K} = 2 \abs{\K} \sin(\theta/2)$ and $\abs{\K} = 4\pi/(3a)$.

The area of the moiré lattice unit cell is
\begin{equation} \label{eq:moire-unitcell}
A_{\text{m.u.c.}} = \frac{(2\pi)^2}{\abs{\b_1^{\text{m}} \cross \b_2^{\text{m}}}} = \frac{\sqrt{3} a^2}{8 \sin[2](\theta/2)}.
\end{equation}

%%%%%%%%%%%%%%%%%%%%%%%%%%%%%%%%%%%%%%%%%%%%%%%%%%%%%%%%%%%%%%%%%%%%%%%%%%%%%%%%%%%%%%%%%%%%%%%%%%
\subsection{Hamiltonian}
%%%%%%%%%%%%%%%%%%%%%%%%%%%%%%%%%%%%%%%%%%%%%%%%%%%%%%%%%%%%%%%%%%%%%%%%%%%%%%%%%%%%%%%%%%%%%%%%%%

Our starting point will be a tight-binding model $ H = H_1 + H_2 + H_{\perp} $, where $H_\ell$ is the hamiltonian of the layer $\ell$, and $H_\perp = V_{12} + V_{12}^\d$ describes the interlayer hybridization. Working with a tight-binding model, it is natural to write this hamiltonian in terms of Bloch waves
\begin{equation} \label{eq:BM-blochwave}
\ket{\psi_{\ell, \k, \alpha}} = \frac{1}{\sqrt{N_\ell}} \sum_{\R_\ell} e^{i \k \vdot (\R_\ell + \vtau_{\ell,\alpha})} \ket{\ell, R_\ell, \alpha},
\end{equation}
where $\ell$ labels the layer, $N_\ell$ is the number of unit cells, $\R_\ell$ are the positions of the underlying Bravais lattice, $\vtau_{\ell,\alpha}$ are the orbital centers (of sublattice $\alpha = A,B$) in the unit cell, and $\ket{\ell,\R_\ell,\alpha}$ are localized Wannier states. In this basis, considering only nearest-neighbor intralayer hopping the hamiltonian of each layer is analogous to equation \ref{eq:monolayer-tight-binding2}, given by
\begin{equation} \label{eq:blg-eachlayer-hamil}
H_\ell(\k) =
\begin{pmatrix}
0 & -t f_\ell(\k) \\
-t f_\ell(\k) & 0
\end{pmatrix},
\end{equation}
with $f_\ell(\k) = \sum_{i=1}^{3} e^{i\k \vdot \vdelta_\ell}$.

In order to describe low energy states, we expand $\k = \pm \K_\ell + \q$, where $\pm \K_\ell$ are the Dirac points of each layer. To first-order in $\q$, the low-energy hamiltonian becomes
\begin{equation} \label{eq:blg-eachlayer-hamil-lowenergy}
H^{\pm \K}(\q) = \hbar v_F \abs{\q}
\begin{pmatrix}
0 & e^{\mp i (\theta_\q - \theta_\ell)} \\
e^{\pm i (\theta_\q - \theta_\ell)} & 0
\end{pmatrix}.
\end{equation}


%%%%%%%%%%%%%%%%%%%%%%%%%%%%%%%%%%%%%%%%%%%%%%%%%%%%%%%%%%%%%%%%%%%%%%%%%%%%%%%%%%%%%%%%%%%%%%%%%%
\subsection{Interlayer hamiltonian}
%%%%%%%%%%%%%%%%%%%%%%%%%%%%%%%%%%%%%%%%%%%%%%%%%%%%%%%%%%%%%%%%%%%%%%%%%%%%%%%%%%%%%%%%%%%%%%%%%%

The tight-binding interlayer Hamiltonian in second quantization reads
\begin{equation} \label{eq:interlayer-hopping}
V_{12} = \sum_{\R_1,\alpha,\R_2,\beta} c_{1,\alpha}^\d t_{12}^{\alpha\beta}(\R_1,\R_2) c_{2,\beta}(\R_2), \quad
t_{12}^{\alpha\beta}(\R_1,\R_2) =
\mel{1,\R_1,\alpha}{V_{12}}{2,\R_2,\beta}.
\end{equation}
Performing the Fourier transformation
\begin{equation} \label{eq:blg-fourier}
c_{\ell,\alpha}^\d(\R_\ell) = \frac{1}{\sqrt{N_\ell}} \sum_{\k_\ell}
e^{-i\k_\ell \vdot (\R_\ell + \vtau_{\ell,\alpha})} c_{\ell,\alpha}^\d(\k_\ell),
\end{equation}
with the sum of $\k_\ell$ over the BZ of layer $\ell$, we have
\begin{equation} \label{eq:interlayer-hopping-kspace}
V_{12} = \sum_{\k_1,\alpha,\k_2,\beta} c_{1,\alpha}^\d(\k_1) T_{12}^{\alpha\beta}(\k_1,\k_2) c_{2,\beta}(\k_2),
\end{equation}
where we define
\begin{equation} \label{eq:interlayer-hopping-Tkspace}
T_{12}^{\alpha\beta}(\k_1,\k_2) =
\frac{1}{\sqrt{N_1 N_2}} \sum_{\R_1,\R_2} e^{-i\k_1\vdot(\R_1+\vtau_{1,\alpha})}
t_{12}^{\alpha\beta}(\R_1,\R_2) e^{i\k_2\vdot(\R_2+\vtau_{2,\beta})}.
\end{equation}

Now we assume that the interlayer hopping $t_{12}^{\alpha\beta}(\R_1,\R_2)$ is only a function of the separation between the centers of the two orbitals, and we write its Fourier transform
\begin{equation} \label{eq:interlayer-hopping-fourier}
t_{12}^{\alpha\beta}(\R_1,\R_2) = t_{12}^{\alpha\beta}(\R_1+\vtau_{1,\alpha}-\R_2-\vtau_{2,\beta}) =
\int \frac{\dd[2]{\p}}{(2\pi)^2} e^{i \p \vdot (\R_1+\vtau_{1,\alpha}-\R_2-\vtau_{2,\beta})} t_{12}^{\alpha\beta}(\p).
\end{equation}

Plugging equation \ref{eq:interlayer-hopping-fourier} in equation \ref{eq:interlayer-hopping-Tkspace}, we have
$$
T_{12}^{\alpha\beta}(\k_1,\k_2) = \frac{1}{\sqrt{N_1 N_2}} \int \frac{\dd[2]{\p}}{(2\pi)^2}
\sum_{\R_1} e^{-i(\k_1-\p)\vdot(\R_1+\vtau_{1,\alpha})} t_{12}^{\alpha\beta}(\p)
\sum_{\R_2} e^{i(\k_2-\p)\vdot(\R_2+\vtau_{2,\beta})} =
$$

$$
= \sqrt{N_1 N_2} \int \frac{\dd[2]{\p}}{(2\pi)^2}
\sum_{\G_1, \G_2} e^{-i\G_1\vdot\vtau_{1,\alpha}} t_{12}^{\alpha\beta}(\p)
e^{i\G_2\vdot\vtau_{2,\beta}} \delta_{\k_1-\p,\G_1} \delta_{\k_2-\p,\G_2},
$$

where we have used the relation $\sum_{\R_\ell} e^{i\k\vdot\R_\ell} = N_\ell \sum_{\G_\ell} \delta_{\k,\G_\ell}$. By elementary Dirac delta properties, \textcolor{red}{substituting $\G_\ell \to -\G_\ell$}, and remembering that $A = A_{u.c.1} N_1 = A_{u.c.2} N_2$, we obtain
\begin{equation} \label{eq:interlayer-hopping-simplified}
T_{12}^{\alpha\beta}(\k_1,\k_2) = \frac{1}{\sqrt{A_{u.c.1} A_{u.c.2}}}
\sum_{\G_1, \G_2} \textcolor{red}{e^{i\G_1\vdot\vtau_{1,\alpha}}} t_{12}^{\alpha\beta}(\p)
\textcolor{red}{e^{-i\G_2\vdot\vtau_{2,\beta}}} \delta_{\k_1+\G_1, \k_2+\G_2},
\end{equation}
where the momentum conservation $ \k_1 + \G_1 = \k_2 + \G_2 $ is known as generalized umklapp condition.

\n

\textcolor{red}{
Now we try to guess the behavior of $t_{12}(\r)$. We often see in the literature the estimate in terms of Slater-Koster parameters $V_{pp\s}$ and $V_{pp\pi}$, where they are assumed to exponentially decay. Making some rough estimates, they conclude that $t_\perp(\p)$ is only a function of $\abs{\p}$ and decays very rapidly as $\abs{\p}$ increases.
}

\begin{figure}[H]
\centering
\includegraphics[width=0.6\linewidth]{fig/tperp.png}
\caption{Fourier transform for the interlayer hopping in tBLG. The vertical dashed line marks the position
of the Dirac point. Taken from \cite{handbook2019}.}
\label{fig:tperp}
\end{figure}

%%%%%%%%%%%%%%%%%%%%%%%%%%%%%%%%%%%%%%%%%%%%%%%%%%%%%%%%%%%%%%%%%%%%%%%%%%%%%%%%%%%%%%%%%%%%%%%%%%
\subsection{\textcolor{red}{Small rotation limit}}
%%%%%%%%%%%%%%%%%%%%%%%%%%%%%%%%%%%%%%%%%%%%%%%%%%%%%%%%%%%%%%%%%%%%%%%%%%%%%%%%%%%%%%%%%%%%%%%%%%

When the rotation angle $\theta \lesssim 10^\circ$ is small enough, we can expand around the Dirac points of each layer $\k_\ell = \K_\ell + \q_\ell$ with $\abs{\q_\ell} \sim \abs{\Delta \K} \ll \abs{\K}$. Approximating $t_\perp(\K_1+\q_1+\G_1) \approx t_\perp(\K_1+\G_1)$, the interlayer coupling becomes
$$
T_{12}^{\alpha\beta}(\q_1,\q_2) = \frac{1}{A_{\text{u.c.}}} \sum_{\G_1,\G_2} e^{i\G_1\vdot\vtau_{1,\alpha}}
t_{12}^{\alpha\beta}(\K_1+\G_1) e^{-i\G_2\vdot\vtau_{2,\beta}}
\delta_{\K_1+\q_1+\G_1,\K_2+\q_2+\G_2}.
$$

As $t_\perp(\p)$ decays rapidly with $\abs{\p}$, we truncate $t_\perp(\p) \approx 0$ for $\abs{\p} > \abs{\K}$. This leaves us only three options for $\G_1$, being $\g_{11} = \0$, $\g_{12} = \b_{12}$, $\g_{13} = -\b_{11}$. These three vectors correspond to the three $\K$ equivalent points on the monolayer BZ. Because of the indifference between layer 1 and 2, the same goes for $\G_2$. Notice that $\abs{\K_1+\g_{1n}} = \abs{\K}$, because they are equivalent $\K$ points. Thus, the only quantity that matters is $t_\perp(\abs{\K})$. We have the term
$$
\delta_{\K_1+\q_1+\G_1,\K_2+\q_2+\G_2} = \delta_{\q_2-\q_1,\K_1-\K_2+\G_1-\G_2}.
$$
We call $\Delta\K = \K_1-\K_2$. Due to $\abs{\q_\ell} \ll \abs{\K}$, we only have that $\q_2-\q_1 = \K_1-\K_2+\G_1-\G_2$ if $\G_1 = \g_{1n}$ and $\G_2 = \g_{2n}$ have the same index $n$. Therefore, we have three possibilities
$$
\q_\text{b} = \K_1 - \K_2 = \abs{\Delta \K} \qty(0, -1),
$$
$$
\q_\text{tr} = (\K_1 - \K_2) + (\g_{12} - \g_{22}) = \abs{\Delta \K} \qty(\frac{\sqrt{3}}{2}, \frac{1}{2}),
$$

$$
\q_\text{tl} = (\K_1 - \K_2) + (\g_{13} - \g_{23}) = \abs{\Delta \K} \qty(-\frac{\sqrt{3}}{2}, \frac{1}{2}),
$$
where $\abs{\Delta \K} = 2 \sin(\theta/2) \abs{\K}$.

The interlayer coupling has only three terms
$$
T_{12}^{\alpha\beta}(\q_1,\q_2) =
T^{\alpha\beta}_{\q_\text{b}} \delta_{\q_1-\q_2, \q_\text{b}} +
T^{\alpha\beta}_{\q_\text{tr}} \delta_{\q_1-\q_2, -\q_\text{tr}} +
T^{\alpha\beta}_{\q_\text{tl}} \delta_{\q_1-\q_2, -\q_\text{tl}},
$$
where $T^{\alpha\beta}_{\q_n} = \frac{t_\perp(\abs{\K})}{A_{\text{u.c.}}} e^{i \g_{1n} \vdot \bm{\tau}_{1\alpha}}
e^{-i \g_{2n} \vdot \bm{\tau}_{2\beta}}$. Writing it in the $A, B$ basis we get
$$
T_{\q_\text{b}} = \frac{t_\perp(\abs{\K})}{A_{\text{u.c.}}}
\begin{pmatrix}
1 & 1 \\
1 & 1
\end{pmatrix},
$$
$$
\textcolor{red}{
T_{\q_\text{tr}} = \frac{t_\perp(\abs{\K})}{A_{\text{u.c.}}} e^{-i \g_{12} \vdot \vtau_0}
\begin{pmatrix}
e^{i\phi} & 1 \\
e^{-i\phi} & e^{i\phi}
\end{pmatrix},
}
$$
$$
\textcolor{red}{
T_{\q_\text{tl}} = \frac{t_\perp(\abs{\K})}{A_{\text{u.c.}}} e^{-i \g_{13} \vdot \vtau_0}
\begin{pmatrix}
e^{-i\phi} & 1 \\
e^{i\phi} & e^{-i\phi}
\end{pmatrix},
}
$$
with $\phi = 2\pi/3$.



%%%%%%%%%%%%%%%%%%%%%%%%%%%%%%%%%%%%%%%%%%%%%%%%%%%%%%%%%%%%%%%%%%%%%%%%%%%%%%%%%%%%%%%%%%%%%%%%%%
\section{AB stacking}
%%%%%%%%%%%%%%%%%%%%%%%%%%%%%%%%%%%%%%%%%%%%%%%%%%%%%%%%%%%%%%%%%%%%%%%%%%%%%%%%%%%%%%%%%%%%%%%%%%

The so-called Bernal or AB stacking of two layers of graphene is such that the atoms of sublattice $A$ from one layer are placed above the atoms of sublattice $B$ from the other layer.
To model the bilayer system in this configuration, we use the monolayer tight-binding hamiltonian (with only the first
nearest neighbors) as a basis and include an interlayer hopping. Indexing the layers with $\ell = 1, 2$, we write
\begin{equation} \label{eq:ab-hamil}
H = H_1 + H_2 + H_{\perp},
\end{equation}
\begin{equation} \label{eq:ab-slg-hamil}
\begin{split}
H_\ell &= -t \sum_{\R} c_{\ell,A}^\d(\R) [c_{\ell,B}(\R) + c_{\ell,B}(\R-\a_1) + c_{\ell,B}(\R-\a_2)] + \hc, \\
H_\perp &= t_{\perp} \sum_{\R} c_{1,A}^\d(\R) c_{2,B}(\R) + \hc,
\end{split}
%\begin{split}
%H_1 &= -t \sum_{\R} c_{1,A}^\d(\R) [c_{1,B}(\R) + c_{1,B}(\R-\a_1) + c_{1,B}(\R-\a_2)] + \hc, \\
%H_2 &= -t \sum_{\R} c_{2,A}^\d(\R) [c_{2,B}(\R) + c_{2,B}(\R-\a_1) + c_{2,B}(\R-\a_2)] + \hc,
%\end{split}
\end{equation}
where $c_{\ell,\alpha}(\R)^\d$ is the creation operator for an electron in a state $\ket{\ell,\R,\alpha}$. Using the discrete Fourier Transform of the creation operators
\begin{equation} \label{eq:ft-creatio}
c_{\ell,\alpha}^\d(\R) = \frac{1}{\sqrt{N}} \sum_{\k\in\text{1BZ}} e^{-i\k\vdot(\R+\bm{\tau}_{\ell,\alpha})} c_{\ell,\alpha}^\d(\k),
\end{equation}
we can rewrite $H = \sum_{\k} \Psi^\d(\k) H(\k) \Psi(\k)$, where $\Psi^\d(\k) = (c_{1,A}^\d(\k) \; c_{1,B}^\d(\k) \; c_{2,A}^\d(\k) \; c_{2,B}^\d(\k))$ and
\begin{equation} \label{eq:ab-momentum_space}
H(\k) =
\begin{pmatrix}
0 & -t f(\k) & 0 & t_\perp \\
-t f^*(\k) & 0 & 0 & 0 \\
0 & 0 & 0 & -t f(\k) \\
t_\perp & 0 & -t f^*(\k) & 0 \\
\end{pmatrix}.
\end{equation}

Diagonalizing $H(\k)$ we obtain the four bands for the AB-stacked bilayer graphene
\begin{equation} \label{eq:ab-four_bands}
E_{\pm,\pm} = \pm t
\sqrt{
\qty(\frac{t_\perp}{2t})^2 +
4 \cos(\frac{\sqrt{3}}{2} d \, k_x) \cos(\frac{3}{2} d \, k_y) + 2 \cos(\sqrt{3} d \, k_x) + 3
}
\; \pm \; \frac{t_\perp}{2}.
\end{equation}






As drawn in Figure \ref{fig:latvec}, we have
\begin{align}
\label{eq:scalarprods0}
\abs{\a_i^{(j)}} &= a, \\
\label{eq:scalarprods1}
\vb{a}_1^{(j)} \vdot \vb{a}_2^{(j)} &= a^2 \cos(60^\circ) = a^2/2, \\
\label{eq:scalarprods2}
\vb{a}_1^{(1)} \vdot \vb{a}_1^{(2)} &= a^2 \cos\theta, \\
\label{eq:scalarprods3}
\vb{a}_1^{(1)} \vdot \vb{a}_2^{(2)} &= a^2 \cos(60^\circ + \theta), \\
\label{eq:scalarprods4}
\vb{a}_1^{(2)} \vdot \vb{a}_2^{(1)} &= a^2 \cos(60^\circ - \theta).
\end{align}

The superlattice vectors $\vb{L}_1$, $\vb{L}_2$ (when the angle is commensurate) are related by a $60^\circ$ rotation. In general, because $\vb{L}_1$ is a point that belongs to the lattices of both layers, it is written by integers $m,n,m',n'$ as
\begin{equation} \label{eq:L1}
\vb{L}_1 = m\vb{a}_1^{(1)} + n\vb{a}_2^{(1)} = m'\vb{a}_1^{(2)} + n'\vb{a}_2^{(2)}.
\end{equation}

Koshino \cite{koshino2012} argues that there is an appropriate choice of lattice vectors $\vb{a}_1^{(1)}, \vb{a}_2^{(1)}, \vb{a}_1^{(2)}, \vb{a}_2^{(2)}$ (satisfying equations \ref{eq:scalarprods0} to \ref{eq:scalarprods4}) such that the indices $(m',n')$ can be made equal to $(n,m)$. By taking the scalar products of equation \ref{eq:L1} with $\vb{a}_1^{(1)}$ and $\vb{a}_1^{(2)}$, we get
$$
\begin{cases}
\; m + n/2 = n \cos\theta + m \cos(60^\circ + \theta); \\
\; m/2 + n = m \cos\theta + n \cos(60^\circ - \theta).
\end{cases}
\Rightarrow
\begin{cases}
\; mn + n^2/2 = n^2 \cos\theta + mn \qty(\frac{\cos\theta}{2}
- \frac{\sqrt{3} \sin\theta}{2}); \\
\; m^2/2 + mn = m^2 \cos\theta + mn \qty(\frac{\cos\theta}{2}
+ \frac{\sqrt{3} \sin\theta}{2}).
\end{cases}
$$

Summing the two equations above gives us
\begin{equation} \label{eq:costheta}
\boxed{\cos\theta = \frac{1}{2} \cdot \frac{m^2 + n^2 + 4mn}{m^2 + n^2 + mn}.}
\end{equation}



\n

\textbf{TEXTO ABAIXO FOI PLAGIADO DO ZOU2018}

Theoretically at small twist angles, there is a well
known “continuum theory” description which yields
well-defined band structures for all twist angles in-
cluding incommensurate ones [32, 33].
The contin-
uum theory reveals many universal features of the band
structure, such as the existence of Dirac crossings be-
tween valence and conduction bands within each val-
ley of the underlying graphene layers. These features
have been benchmarked against tight-binding calcula-
tions on commensurate structures [34–37]. They are
also nicely consistent with experiments at twist angles
larger than the magic angles (where correlation effects
are expected to be weaker, and band theory predic-
tions can be reasonably compared with experiment).
In particular, Cao et al showed that at a twist angle of
about 1.8 degrees the Landau fan structure near charge
neutrality is exactly what is expected from the Dirac
points predicted by the continuum theory [38]. Despite
its success for qualitative universal aspects, quantita-
tively the continuum theory yields a very small value
compared to experiments for the gap separating the
nearly flat bands from other bands. This discrepancy is
believed to be reduced once effects of lattice relaxation
and electron interactions are included. Formally these
additional effects can be included phenomenologically
in the continuum model by modifying its parameters
away from those estimated microscopically [6, 21].
Apart from translational symmetry, the approxima-
tions involved in the continuum theory build in a num-
ber of other point group symmetries which are not fully
present in any commensurate structure. These include
a C6 rotation symmetry, and a valley Uv(1) associated
with separate conservation of electrons associated with
each valley.
This symmetry structure of the contin-
uum theory is essential in protecting the Dirac points
of valley filtered bands. Specifically, on top of the val-
ley Uv(1) (needed to define separate bands within each
valley) a C2T = C3
6T symmetry is able to protect the
Dirac points from acquiring a gap.
Even restricting
to commensurate structures with translation symme-
try, Uv(1), and maybe even C2T , are not both exact
microscopic symmetries.

In the older literature it was appreciated that at
small twist angles the extra symmetries of the contin-
uum theory are excellent approximations [34–36, 39].
Furthermore it was understood that there is essentially
no difference between incommensurate and commen-
surate structures, or between distinct commensurate
structures with different exact microscopic symmetries.
These issues have re-emerged in recent discussions of
TBG, and have led to some confusion. We therefore
carefully review and collect together some pertinent
facts about different commensurate structures, their
relationship to the continuum theory, and the implica-
tions for a description of small angle (possibly incom-
mensurate) TBG.

The most fundamental aspect of our previous dis-
cussion of the band structure is the existence of an ob-
struction to constructing well localized Wannier func-
tions transforming naturally under all symmetry oper-
ations. The obstruction relies strongly on the presence
of symmetries that are not exact microscopic symme-
tries.
Why then should we worry about it?
Let us
therefore review the tight logic that forces us to con-
front it. As reviewed above, it is a robust feature of
both theory and experiment that to excellent accuracy
there is a good valley Uv(1) symmetry and that within
each valley there are Dirac band crossings (down to en-
ergy scales currently accessible in experiments). The
robustness of the Dirac crossings within each valley
suggests that it is a symmetry protected feature of the
band structure. The natural protecting symmetry then
is C2T as is seen explicitly in the continuum theory.
For a general small-angle, incommensurate TBG struc-
ture, the C2 symmetry—like translations itself—is not
an exact symmetry, but it must be excellent enough
to give the Dirac cones. This then forces us to study
systems which have translations, valley Uv(1), and C6
as good symmetries. However the implementation of
all these symmetries in the band structure leads to a
Wannier obstruction.




%%%%%%%%%%%%%%%%%%%%%%%%%%%%%%%%%%%%%%%%%%%%%%%%%%%%%%%%%%%%%%%%%%%%%%%%%%%%%%%%%%%%%%%%%%%%%%%%%%
%%%%%%%%%%%%%%%%%%%%%%%%%%%%%%%%%%%%%%%%%%%%%%%%%%%%%%%%%%%%%%%%%%%%%%%%%%%%%%%%%%%%%%%%%%%%%%%%%%


%%%%%%%%%%%%%%%%%%%%%%%%%%%%%%%% COMMENT THIS TO COMPILE main.tex %%%%%%%%%%%%%%%%%%%%%%%%%%%%%%%%
%%-----
%% Referências bibliográficas
%%-----
\addcontentsline{toc}{chapter}{\bibname}
%\bibliographystyle{abntex2-num}
\bibliography{citations}
\bibliographystyle{ieeetr}
\end{document}
%%%%%%%%%%%%%%%%%%%%%%%%%%%%%%%% COMMENT THIS TO COMPILE main.tex %%%%%%%%%%%%%%%%%%%%%%%%%%%%%%%%
