\documentclass[a4paper,10pt]{article}
\documentclass[a4paper,12pt]{report}
\usepackage[english]{babel}
\usepackage[left=2cm,right=2cm,top=2cm,bottom=2cm]{geometry}
%\usepackage{mathtools}
\usepackage{amsthm}     % for definitions and theorems
\usepackage[many]{tcolorbox}    % boxes around definitions and theorems
%\usepackage{amsmath}
%\usepackage{nccmath}
\usepackage{amssymb}    % \ltimes
\usepackage{etoolbox}   % for start of Chapter
%\usepackage{amsfonts}
\usepackage{physics}    % for all Physics related
%\usepackage{dsfont}
%\usepackage{mathrsfs}

\usepackage{titling}
\usepackage{indentfirst}

\usepackage{bm}
\usepackage[dvipsnames]{xcolor}
\usepackage{cancel}

\usepackage{xurl}
\usepackage[colorlinks=true]{hyperref}

\usepackage{float}
\usepackage{graphicx}
\usepackage{subcaption}
%\usepackage{tikz}

\usepackage{ctable}     % tabelas
\renewcommand{\P}{\phantom{+}}  % empty space to indent things
\usepackage{multirow}
\usepackage{tabulary}

%%%%%%%%%%%%%%%%%%%%%%%%%%%%%%%%%%%%%%%%%%%%%%%%%%%

\newcommand{\eps}{\epsilon}
\newcommand{\vphi}{\varphi}
\newcommand{\cte}{\text{cte}}

\newcommand{\N}{{\mathbb{N}}}
\newcommand{\Z}{{\mathbb{Z}}}
%\newcommand{\Q}{{\mathbb{Q}}}
\newcommand{\C}{{\mathbb{C}}}
\renewcommand{\S}{{\hat{S}}}
%\renewcommand{\H}{\s{H}}

\renewcommand{\a}{{\vb{a}}}
\renewcommand{\b}{{\vb{b}}}
\renewcommand{\d}{{\dagger}}
\newcommand{\up}{{\uparrow}}
\newcommand{\down}{{\downarrow}}
\newcommand{\hc}{{\text{h.c.}}}

\newcommand{\ihat}{\bm{\hat{\imath}}}
\newcommand{\jhat}{\bm{\hat{\jmath}}}
\newcommand{\khat}{\bm{\hat{k}}}

\newcommand{\0}{{\vb{0}}}
%\newcommand{\1}{\mathds{1}}
\newcommand{\E}{{\vb{E}}}
\newcommand{\B}{{\vb{B}}}
\renewcommand{\u}{{\vb{u}}}
\renewcommand{\v}{{\vb{v}}}
\renewcommand{\r}{{\vb{r}}}
\newcommand{\R}{{\vb{R}}}
\newcommand{\Q}{{\vb{Q}}}
\newcommand{\G}{{\vb{G}}}
\newcommand{\g}{{\vb{g}}}
\renewcommand{\k}{{\vb{k}}}
\newcommand{\K}{{\vb{K}}}
\newcommand{\p}{{\vb{p}}}
\newcommand{\q}{{\vb{q}}}
\newcommand{\F}{{\vb{F}}}
\renewcommand{\t}{{\vb{t}}}
\newcommand{\vtau}{{\bm{\tau}}}
\newcommand{\vdelta}{{\bm{\delta}}}

\newcommand{\s}{\sigma}
%\newcommand{\prodint}[2]{\left\langle #1 , #2 \right\rangle}
\newcommand{\cc}[1]{\overline{#1}}
\newcommand{\Eval}[3]{\eval{\left( #1 \right)}_{#2}^{#3}}
\newcommand{\sg}[2]{\{ #1 \mid #2 \}}

\newcommand{\unit}[1]{\; \mathrm{#1}}

\newcommand{\n}{\medskip}
\newcommand{\e}{\quad \mathrm{and} \quad}
\newcommand{\ou}{\quad \mathrm{or} \quad}
\newcommand{\virg}{\, , \;}
\newcommand{\ptodo}{\forall \,}
\renewcommand{\implies}{\; \Rightarrow \;}
%\newcommand{\eqname}[1]{\tag*{#1}} % Tag equation with name

\setlength{\droptitle}{-7em}

\makeatletter
\patchcmd{\chapter}{\if@openright\cleardoublepage\else\clearpage\fi}{}{}{}  % start 'Chapter' at the same page. needs package etoolbox
\makeatother

%% Theorems, definitions, proofs
\theoremstyle{definition}

\newtheorem{definition}{Definition}[section]
\tcolorboxenvironment{definition}{
  colback=blue!5!white,
  boxrule=0pt,
  boxsep=1pt,
  left=2pt,right=2pt,top=2pt,bottom=2pt,
  oversize=2pt,
  sharp corners,
  before skip=\topsep,
  after skip=\topsep,
}

\newtheorem{theorem}{Theorem}[section]
\tcolorboxenvironment{theorem}{
  colback=blue!5!white,
  boxrule=0pt,
  boxsep=1pt,
  left=2pt,right=2pt,top=2pt,bottom=2pt,
  oversize=2pt,
  sharp corners,
  before skip=\topsep,
  after skip=\topsep,
}


%\documentclass[../main.tex]{subfiles}
%\graphicspath{{\subfix{../fig/}}}

\begin{document}

\begin{figure}[H]
\centering
\includegraphics[width=0.2\textwidth]{fig/latvec.png}
\caption{Commensurate angle case and lattice vectors. Taken from \cite{koshino}}
\label{fig:latvec}
\end{figure}

As drawn in Figure \ref{fig:latvec}, we have
\begin{align}
\label{eq:scalarprods1}
\vb{a}_1^{(1)} \vdot \vb{a}_2^{(1)} &= a^2 \cos(60^\circ) = a^2/2; \\
\label{eq:scalarprods2}
\vb{a}_1^{(1)} \vdot \vb{a}_1^{(2)} &= a^2 \cos\theta; \\
\label{eq:scalarprods3}
\vb{a}_1^{(1)} \vdot \vb{a}_2^{(2)} &= a^2 \cos(60^\circ + \theta); \\
\label{eq:scalarprods4}
\vb{a}_1^{(2)} \vdot \vb{a}_2^{(1)} &= a^2 \cos(60^\circ - \theta).
\end{align}

The superlattice vectors $\vb{L}_1$, $\vb{L}_2$ (when the angle is commensurate) are related by a $60^\circ$ rotation. In general, because $\vb{L}_1$ is a point that belongs to the lattices of both layers, it is written by integers $m,n,m',n'$ as
\begin{equation} \label{eq:L1}
\vb{L}_1 = m\vb{a}_1^{(1)} + n\vb{a}_2^{(1)} = m'\vb{a}_1^{(2)} + n'\vb{a}_2^{(2)}.
\end{equation}

Koshino \cite{koshino} argues that there is an appropriate choice of lattice vectors $\vb{a}_1^{(1)}, \vb{a}_2^{(1)}, \vb{a}_1^{(2)}, \vb{a}_2^{(2)}$ (satisfying equations \ref{eq:scalarprods1} to \ref{eq:scalarprods4}) such that the indices $(m',n')$ can be made equal to $(n,m)$. If this is true, then by taking the scalar products of equation \ref{eq:L1} with $\vb{a}_1^{(1)}$ and $\vb{a}_1^{(2)}$, we get
$$
\begin{cases}
\; m + n/2 = n \cos\theta + m \cos(60^\circ + \theta); \\
\; m/2 + n = m \cos\theta + n \cos(60^\circ - \theta).
\end{cases}
\Rightarrow
\begin{cases}
\; mn + n^2/2 = n^2 \cos\theta + mn \qty(\frac{\cos\theta}{2}
- \frac{\sqrt{3} \sin\theta}{2}); \\
\; m^2/2 + mn = m^2 \cos\theta + mn \qty(\frac{\cos\theta}{2}
+ \frac{\sqrt{3} \sin\theta}{2}).
\end{cases}
$$

Summing the two equations above gives us
$$
\boxed{\cos\theta = \frac{1}{2} \cdot \frac{m^2 + n^2 + 4mn}{m^2 + n^2 + mn}.}
$$



%%-----
%% Referências bibliográficas
%%-----
\addcontentsline{toc}{chapter}{\bibname}
%\bibliographystyle{abntex2-num}
\bibliography{citations}
\bibliographystyle{ieeetr}


\end{document}
