%%%%%%%%%%%%%%%%%%%%%%%%%%%%%%%%% COMMENT THIS TO COMPILE main.tex %%%%%%%%%%%%%%%%%%%%%%%%%%%%%%%%
%\documentclass[a4paper,12pt]{report}
\usepackage[english]{babel}
\usepackage[left=2cm,right=2cm,top=2cm,bottom=2cm]{geometry}
%\usepackage{mathtools}
\usepackage{amsthm}     % for definitions and theorems
\usepackage[many]{tcolorbox}    % boxes around definitions and theorems
%\usepackage{amsmath}
%\usepackage{nccmath}
\usepackage{amssymb}    % \ltimes
\usepackage{etoolbox}   % for start of Chapter
%\usepackage{amsfonts}
\usepackage{physics}    % for all Physics related
%\usepackage{dsfont}
%\usepackage{mathrsfs}

\usepackage{titling}
\usepackage{indentfirst}

\usepackage{bm}
\usepackage[dvipsnames]{xcolor}
\usepackage{cancel}

\usepackage{xurl}
\usepackage[colorlinks=true]{hyperref}

\usepackage{float}
\usepackage{graphicx}
\usepackage{subcaption}
%\usepackage{tikz}

\usepackage{ctable}     % tabelas
\renewcommand{\P}{\phantom{+}}  % empty space to indent things
\usepackage{multirow}
\usepackage{tabulary}

%%%%%%%%%%%%%%%%%%%%%%%%%%%%%%%%%%%%%%%%%%%%%%%%%%%

\newcommand{\eps}{\epsilon}
\newcommand{\vphi}{\varphi}
\newcommand{\cte}{\text{cte}}

\newcommand{\N}{{\mathbb{N}}}
\newcommand{\Z}{{\mathbb{Z}}}
%\newcommand{\Q}{{\mathbb{Q}}}
\newcommand{\C}{{\mathbb{C}}}
\renewcommand{\S}{{\hat{S}}}
%\renewcommand{\H}{\s{H}}

\renewcommand{\a}{{\vb{a}}}
\renewcommand{\b}{{\vb{b}}}
\renewcommand{\d}{{\dagger}}
\newcommand{\up}{{\uparrow}}
\newcommand{\down}{{\downarrow}}
\newcommand{\hc}{{\text{h.c.}}}

\newcommand{\ihat}{\bm{\hat{\imath}}}
\newcommand{\jhat}{\bm{\hat{\jmath}}}
\newcommand{\khat}{\bm{\hat{k}}}

\newcommand{\0}{{\vb{0}}}
%\newcommand{\1}{\mathds{1}}
\newcommand{\E}{{\vb{E}}}
\newcommand{\B}{{\vb{B}}}
\renewcommand{\u}{{\vb{u}}}
\renewcommand{\v}{{\vb{v}}}
\renewcommand{\r}{{\vb{r}}}
\newcommand{\R}{{\vb{R}}}
\newcommand{\Q}{{\vb{Q}}}
\newcommand{\G}{{\vb{G}}}
\newcommand{\g}{{\vb{g}}}
\renewcommand{\k}{{\vb{k}}}
\newcommand{\K}{{\vb{K}}}
\newcommand{\p}{{\vb{p}}}
\newcommand{\q}{{\vb{q}}}
\newcommand{\F}{{\vb{F}}}
\renewcommand{\t}{{\vb{t}}}
\newcommand{\vtau}{{\bm{\tau}}}
\newcommand{\vdelta}{{\bm{\delta}}}

\newcommand{\s}{\sigma}
%\newcommand{\prodint}[2]{\left\langle #1 , #2 \right\rangle}
\newcommand{\cc}[1]{\overline{#1}}
\newcommand{\Eval}[3]{\eval{\left( #1 \right)}_{#2}^{#3}}
\newcommand{\sg}[2]{\{ #1 \mid #2 \}}

\newcommand{\unit}[1]{\; \mathrm{#1}}

\newcommand{\n}{\medskip}
\newcommand{\e}{\quad \mathrm{and} \quad}
\newcommand{\ou}{\quad \mathrm{or} \quad}
\newcommand{\virg}{\, , \;}
\newcommand{\ptodo}{\forall \,}
\renewcommand{\implies}{\; \Rightarrow \;}
%\newcommand{\eqname}[1]{\tag*{#1}} % Tag equation with name

\setlength{\droptitle}{-7em}

\makeatletter
\patchcmd{\chapter}{\if@openright\cleardoublepage\else\clearpage\fi}{}{}{}  % start 'Chapter' at the same page. needs package etoolbox
\makeatother

%% Theorems, definitions, proofs
\theoremstyle{definition}

\newtheorem{definition}{Definition}[section]
\tcolorboxenvironment{definition}{
  colback=blue!5!white,
  boxrule=0pt,
  boxsep=1pt,
  left=2pt,right=2pt,top=2pt,bottom=2pt,
  oversize=2pt,
  sharp corners,
  before skip=\topsep,
  after skip=\topsep,
}

\newtheorem{theorem}{Theorem}[section]
\tcolorboxenvironment{theorem}{
  colback=blue!5!white,
  boxrule=0pt,
  boxsep=1pt,
  left=2pt,right=2pt,top=2pt,bottom=2pt,
  oversize=2pt,
  sharp corners,
  before skip=\topsep,
  after skip=\topsep,
}

%\begin{document}
%%%%%%%%%%%%%%%%%%%%%%%%%%%%%%%%% COMMENT THIS TO COMPILE main.tex %%%%%%%%%%%%%%%%%%%%%%%%%%%%%%%%


%%%%%%%%%%%%%%%%%%%%%%%%%%%%%%%%%%%%%%%%%%%%%%%%%%%%%%%%%%%%%%%%%%%%%%%%%%%%%%%%%%%%%%%%%%%%%%%%%%
\chapter{Introduction} \label{ch:intro}
%%%%%%%%%%%%%%%%%%%%%%%%%%%%%%%%%%%%%%%%%%%%%%%%%%%%%%%%%%%%%%%%%%%%%%%%%%%%%%%%%%%%%%%%%%%%%%%%%%

Carbon is undoubtedly one of the most essential elements for the formation of life, largely due to its remarkable versatility in bonding, which enables the formation of a vast array of complex compounds. Among materials composed solely of carbon atoms, graphite is one of which we are very familiar since elementary school, as it used in pencils for drawing. Graphite is an allotrope of carbon, consisting of multiple layers stacked together and held by weak van der Waals forces. When writing with a pencil, some graphite layers transfer onto the paper, allowing for drawing and writing.

In 2004, Andre Geim, Konstantin Novoselov, et. al. \cite{novoselov_2004} successfully isolated a single sheet of graphite using the ``sticky tape method'', allowing them to study its properties. This single atomic layer, known as graphene, was the first two-dimensional (2D) material ever discovered. Graphene exhibits extraordinary properties, including exceptional electrical conductivity, mechanical strength, flexibility, transparency, and thermal conductivity. Its unique combination of properties makes it a promising candidate for a wide range of applications, from electronics and energy storage to composite materials and biomedical devices. The discovery of graphene marked a breakthrough, inaugurating the field of 2D materials and earning Geim and Novoselov the Nobel Prize in Physics in 2010.

When we vertically stack a few layers of 2D materials, the system is held together by weak van der Waals forces and is known as a van der Waals heterostructure. These heterostructures represent a promising new avenue for engineering the properties of 2D materials. While the properties of single-layer materials, such as graphene, are now well understood, predicting the behavior of few-layer heterostructures remains challenging. <++>

%%%%%%%%%%%%%%%%%%%%%%%%%%%%%%%%%%%%%%%%%%%%%%%%%%%%%%%%%%%%%%%%%%%%%%%%%%%%%%%%%%%%%%%%%%%%%%%%%%
\section{Monolayer}
%%%%%%%%%%%%%%%%%%%%%%%%%%%%%%%%%%%%%%%%%%%%%%%%%%%%%%%%%%%%%%%%%%%%%%%%%%%%%%%%%%%%%%%%%%%%%%%%%%

Throughout our work, $\hbar = 1$.

\n

Graphene is an allotrope of graphite and consists of a honeycomb lattice of carbon atoms linked in a $sp^2$ hybridization with an average distance $a = 0.246 \unit{nm}$, where three valence electrons from each carbon atom form $\sigma$ bonds and the last valence electron is located on a $\pi$ orbital and is the one that predominantly matters for the electronic properties of the material.

\n

Our convention is zig-zag in horizontal direction and arm-chair in vertical direction.

\begin{figure}[H]
\centering
\includegraphics[width=0.4\linewidth]{fig/graphene-lattice_vectors2.png}
\label{fig:graphene-lattice_vectors}
\caption{\textcolor{red}{Graphene lattice vectors. Taken from \cite{handbook2019}}}
\end{figure}

The lattice vectors are $\vb{a}_1 = a \qty(\frac{1}{2}, \frac{\sqrt{3}}{2})$, $\vb{a}_2 = a \qty(-\frac{1}{2}, \frac{\sqrt{3}}{2})$. The unit cell area is $ A_1 = \abs{\a_1 \times \a_2} = \frac{\sqrt{3}}{2} a^2 $.

The Brillouin zone is
\begin{figure}[H]
\centering
\includegraphics[width=0.3\linewidth]{fig/brillouin-zone-monolayer.png}
\caption{\textcolor{red}{Graphene Brillouin Zone. Taken from \cite{handbook2019}}}
\label{fig:brillouin-zone-monolayer}
\end{figure}

Using the convention $\a_3 = \vu{z}$, the momentum lattice vectors are
\begin{equation} \label{eq:monolayer-bvecs}
\b_1 = \frac{2\pi}{A_1} \a_2 \times \a_3 = \frac{4\pi}{\sqrt{3} a } \qty(\frac{\sqrt{3}}{2}, \frac{1}{2}), \quad
\b_2 = \frac{2\pi}{A_1} \a_3 \times \a_1 = \frac{4\pi}{\sqrt{3} a } \qty(-\frac{\sqrt{3}}{2}, \frac{1}{2}).
\end{equation}

The two main Dirac points are defined as
\begin{equation} \label{eq:monolayer-dirac-points}
\K = \frac{4\pi}{3a} (1, 0) , \quad \K' = -\K.
\end{equation}



We write a tight-binding hamiltonian with only nearest-neighbor hopping
\begin{equation} \label{eq:monolayer-tight-binding}
H = -t \sum_{\R} c^\d_B(\R) \qty(c_A(\R) + c_A(\R-\a_1) + c_A(\R-\a_2)) + \hc,
\end{equation}
where the $\R$ sum runs through all the associated Bravais triangular lattice. Applying the Fourier transforms
\begin{equation} \label{eq:monolayer-fourier}
c_{\alpha}^\d(\R) = \frac{1}{\sqrt{N}} \sum_{\k \in \text{BZ}} e^{-i \k \vdot \r_i} c_A^\d(\k),
\end{equation}
we get
\begin{equation} \label{eq:monolayer-tight-binding2}
H = \sum_{\k}
\begin{pmatrix}
c_A^\d(\k) & c_B^\d(\k)
\end{pmatrix}
\begin{pmatrix}
0 & -t f(\k) \\
-t f^*(\k) & 0
\end{pmatrix}
\begin{pmatrix}
c_A^\d(\k) \\ c_B^\d(\k)
\end{pmatrix},
\end{equation}
with
\begin{equation} \label{eq:monolayer-fk}
f(\k) = \sum_{\nu} e^{i \k \vdot \bm{\delta}_\nu} =
e^{idk_y} + 2 e^{-\frac{idk_y}{2}} \cos(\frac{\sqrt{3}}{2} dk_x),
\end{equation}
where $\bm{\delta}_1 = (\a_1 + \a_2)/3 = d (0, 1)$, $\bm{\delta}_2 = (-2\a_1 + \a_2)/3 = d(-\frac{\sqrt{3}}{2}, -\frac{1}{2})$, $\bm{\delta}_3 = (\a_1 - 2\a_2)/3 = d (\frac{\sqrt{3}}{2}, -\frac{1}{2})$ are the vectors that connect to the three nearest neighboring sites, and $d = a/\sqrt{3}$.

\n

The eigenenergies of hamiltonian \ref{eq:monolayer-tight-binding2} are $E_\pm(\k) = \pm t \abs{f(\k)}$. Notice that the points $\K$ and $\K'=-\K$ are roots of $f$, i.e., $f(\pm\K) = 0$. These are the so-called Dirac points, around which the dispersion relation are approximately linear $E(\K + \q) = v_F \abs{\q} + O(\abs{\q}/\abs{\K})^2$.


%%%%%%%%%%%%%%%%%%%%%%%%%%%%%%%%%%%%%%%%%%%%%%%%%%%%%%%%%%%%%%%%%%%%%%%%%%%%%%%%%%%%%%%%%%%%%%%%%%
%%%%%%%%%%%%%%%%%%%%%%%%%%%%%%%%%%%%%%%%%%%%%%%%%%%%%%%%%%%%%%%%%%%%%%%%%%%%%%%%%%%%%%%%%%%%%%%%%%


%%%%%%%%%%%%%%%%%%%%%%%%%%%%%%%%% COMMENT THIS TO COMPILE main.tex %%%%%%%%%%%%%%%%%%%%%%%%%%%%%%%%
%%%-----
%%% Referências bibliográficas
%%%-----
%\addcontentsline{toc}{chapter}{\bibname}
%%\bibliographystyle{abntex2-num}
%\bibliography{citations}
%\bibliographystyle{ieeetr}
%\end{document}
%%%%%%%%%%%%%%%%%%%%%%%%%%%%%%%%% COMMENT THIS TO COMPILE main.tex %%%%%%%%%%%%%%%%%%%%%%%%%%%%%%%%
