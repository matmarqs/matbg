\documentclass[a4paper,10pt]{article}
\documentclass[a4paper,12pt]{report}
\usepackage[english]{babel}
\usepackage[left=2cm,right=2cm,top=2cm,bottom=2cm]{geometry}
%\usepackage{mathtools}
\usepackage{amsthm}     % for definitions and theorems
\usepackage[many]{tcolorbox}    % boxes around definitions and theorems
%\usepackage{amsmath}
%\usepackage{nccmath}
\usepackage{amssymb}    % \ltimes
\usepackage{etoolbox}   % for start of Chapter
%\usepackage{amsfonts}
\usepackage{physics}    % for all Physics related
\usepackage{dsfont}     % for the identity matrix symbol \1
%\usepackage{mathrsfs}

\usepackage{titling}
\usepackage{indentfirst}

\usepackage{bm}
\usepackage[dvipsnames]{xcolor}
\usepackage{cancel}

\usepackage{xurl}
\usepackage[colorlinks=true]{hyperref}

\usepackage{float}
\usepackage{graphicx}
\usepackage{subcaption}
%\usepackage{tikz}

\usepackage{ctable}     % tabelas
\renewcommand{\P}{\phantom{+}}  % empty space to indent things
\usepackage{multirow}
\usepackage{tabulary}

%%%%%%%%%%%%%%%%%%%%%%%%%%%%%%%%%%%%%%%%%%%%%%%%%%%

\newcommand{\eps}{\epsilon}
\newcommand{\vphi}{\varphi}
\newcommand{\cte}{\text{cte}}

\newcommand{\N}{{\mathbb{N}}}
\newcommand{\Z}{{\mathbb{Z}}}
%\newcommand{\Q}{{\mathbb{Q}}}
\newcommand{\C}{{\mathbb{C}}}
\renewcommand{\S}{{\hat{S}}}
%\renewcommand{\H}{\s{H}}

\renewcommand{\a}{{\vb{a}}}
\renewcommand{\b}{{\vb{b}}}
\renewcommand{\d}{{\dagger}}
\newcommand{\up}{{\uparrow}}
\newcommand{\down}{{\downarrow}}
\newcommand{\hc}{{\text{h.c.}}}

\newcommand{\ihat}{\bm{\hat{\imath}}}
\newcommand{\jhat}{\bm{\hat{\jmath}}}
\newcommand{\khat}{\bm{\hat{k}}}

\newcommand{\0}{{\vb{0}}}
\newcommand{\1}{\mathds{1}}
\newcommand{\E}{{\vb{E}}}
\newcommand{\B}{{\vb{B}}}
\renewcommand{\u}{{\vb{u}}}
\renewcommand{\v}{{\vb{v}}}
\renewcommand{\r}{{\vb{r}}}
\newcommand{\R}{{\vb{R}}}
\newcommand{\Q}{{\vb{Q}}}
\newcommand{\G}{{\vb{G}}}
\newcommand{\g}{{\vb{g}}}
\renewcommand{\k}{{\vb{k}}}
\newcommand{\K}{{\vb{K}}}
\newcommand{\p}{{\vb{p}}}
\newcommand{\q}{{\vb{q}}}
\newcommand{\F}{{\vb{F}}}
\renewcommand{\t}{{\vb{t}}}
\newcommand{\vtau}{{\bm{\tau}}}
\newcommand{\vdelta}{{\bm{\delta}}}

% COLORED SYMMETRY ELEMENTS
\newcommand{\Ct}{{\textcolor{Cyan}{C_3}}}
\newcommand{\Ctn}[1]{{\textcolor{Cyan}{C_3^{\textcolor{black}{#1}}}}}
\newcommand{\Cs}{{\textcolor{ForestGreen}{C_6}}}
\newcommand{\Csn}[1]{{\textcolor{ForestGreen}{C_6^{\textcolor{black}{#1}}}}}
\newcommand{\sd}{{\textcolor{RoyalBlue}{\sigma_d}}}
\newcommand{\sdn}[1]{{\textcolor{RoyalBlue}{\sigma_d^{\textcolor{black}{#1}}}}}
\newcommand{\sdp}{{\textcolor{RoyalBlue}{\sigma_d'}}}
\newcommand{\sdpp}{{\textcolor{RoyalBlue}{\sigma_d''}}}
\newcommand{\sv}{{\textcolor{Orange}{\sigma_v}}}
\newcommand{\svn}[1]{{\textcolor{Orange}{\sigma_v^{\textcolor{black}{#1}}}}}
\newcommand{\svp}{{\textcolor{Orange}{\sigma_v'}}}
\newcommand{\svpp}{{\textcolor{Orange}{\sigma_v''}}}

\newcommand{\s}{\sigma}
%\newcommand{\prodint}[2]{\left\langle #1 , #2 \right\rangle}
\newcommand{\cc}[1]{\overline{#1}}
\newcommand{\Eval}[3]{\eval{\left( #1 \right)}_{#2}^{#3}}
\newcommand{\sg}[2]{\{ #1 \mid #2 \}}

\newcommand{\unit}[1]{\; \mathrm{#1}}

\newcommand{\n}{\medskip}
\newcommand{\e}{\quad \mathrm{and} \quad}
\newcommand{\ou}{\quad \mathrm{or} \quad}
\newcommand{\virg}{\, , \;}
\newcommand{\ptodo}{\forall \,}
\renewcommand{\implies}{\; \Rightarrow \;}
%\newcommand{\eqname}[1]{\tag*{#1}} % Tag equation with name

\setlength{\droptitle}{-7em}

\makeatletter
\patchcmd{\chapter}{\if@openright\cleardoublepage\else\clearpage\fi}{}{}{}  % start 'Chapter' at the same page. needs package etoolbox
\makeatother

%% Theorems, definitions, proofs
\theoremstyle{definition}

\newtheorem{definition}{Definition}[section]
\tcolorboxenvironment{definition}{
  colback=blue!5!white,
  boxrule=0pt,
  boxsep=1pt,
  left=2pt,right=2pt,top=2pt,bottom=2pt,
  oversize=2pt,
  sharp corners,
  before skip=\topsep,
  after skip=\topsep,
}

\newtheorem{theorem}{Theorem}[section]
\tcolorboxenvironment{theorem}{
  colback=blue!5!white,
  boxrule=0pt,
  boxsep=1pt,
  left=2pt,right=2pt,top=2pt,bottom=2pt,
  oversize=2pt,
  sharp corners,
  before skip=\topsep,
  after skip=\topsep,
}


%\documentclass[../main.tex]{subfiles}
%\graphicspath{{\subfix{../fig/}}}

%\usepackage{breqn}

\begin{document}

\section{AB stacking}

The so-called Bernal or AB stacking of two layers of graphene is such that the atoms of sublattice $A$ from one layer are placed above the atoms of sublattice $B$ from the other layer.
To model the bilayer system in this configuration, we use the monolayer tight-binding hamiltonian (with only the first
nearest neighbors) as a basis and include an interlayer hopping. Indexing the layers with $\ell = 1, 2$, we write
\begin{equation} \label{eq:ab-hamil}
H = H_1 + H_2 + H_{\perp},
\end{equation}
\begin{equation} \label{eq:ab-slg-hamil}
\begin{split}
H_\ell &= -t \sum_{\R} c_{\ell,A}^\d(\R) [c_{\ell,B}(\R) + c_{\ell,B}(\R-\a_1) + c_{\ell,B}(\R-\a_2)] + \hc, \\
H_\perp &= t_{\perp} \sum_{\R} c_{1,A}^\d(\R) c_{2,B}(\R) + \hc,
\end{split}
%\begin{split}
%H_1 &= -t \sum_{\R} c_{1,A}^\d(\R) [c_{1,B}(\R) + c_{1,B}(\R-\a_1) + c_{1,B}(\R-\a_2)] + \hc, \\
%H_2 &= -t \sum_{\R} c_{2,A}^\d(\R) [c_{2,B}(\R) + c_{2,B}(\R-\a_1) + c_{2,B}(\R-\a_2)] + \hc,
%\end{split}
\end{equation}
where $c_{\ell,\alpha}(\R)^\d$ is the creation operator for an electron in a state $\ket{\ell,\R,\alpha}$. Using the discrete Fourier Transform of the creation operators
\begin{equation} \label{eq:ft-creatio}
c_{\ell,\alpha}^\d(\R) = \frac{1}{\sqrt{N}} \sum_{\k\in\text{1BZ}} e^{-i\k\vdot(\R+\bm{\tau}_{\ell,\alpha})} c_{\ell,\alpha}^\d(\k),
\end{equation}
we can rewrite $H = \sum_{\k} \Psi^\d(\k) H(\k) \Psi(\k)$, where $\Psi^\d(\k) = (c_{1,A}^\d(\k) \; c_{1,B}^\d(\k) \; c_{2,A}^\d(\k) \; c_{2,B}^\d(\k))$ and
\begin{equation} \label{eq:ab-momentum_space}
H(\k) =
\begin{pmatrix}
0 & -t f(\k) & 0 & t_\perp \\
-t f^*(\k) & 0 & 0 & 0 \\
0 & 0 & 0 & -t f(\k) \\
t_\perp & 0 & -t f^*(\k) & 0 \\
\end{pmatrix}.
\end{equation}

Diagonalizing $H(\k)$ we obtain the four bands for the AB-stacked bilayer graphene
\begin{equation} \label{eq:ab-four_bands}
E_{\pm,\pm} = \pm t
\sqrt{
\qty(\frac{t_\perp}{2t})^2 +
4 \cos(\frac{\sqrt{3}}{2} d \, k_x) \cos(\frac{3}{2} d \, k_y) + 2 \cos(\sqrt{3} d \, k_x) + 3
}
\; \pm \; \frac{t_\perp}{2}.
\end{equation}


%%-----
%% Referências bibliográficas
%%-----
\addcontentsline{toc}{chapter}{\bibname}
%\bibliographystyle{abntex2-num}
\bibliography{citations}
\bibliographystyle{ieeetr}


\end{document}
