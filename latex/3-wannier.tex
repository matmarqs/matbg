%%%%%%%%%%%%%%%%%%%%%%%%%%%%%%% COMMENT THIS TO COMPILE main.tex %%%%%%%%%%%%%%%%%%%%%%%%%%%%%%%%
\documentclass[a4paper,12pt]{report}
\usepackage[english]{babel}
\usepackage[left=2cm,right=2cm,top=2cm,bottom=2cm]{geometry}
%\usepackage{mathtools}
\usepackage{amsthm}     % for definitions and theorems
\usepackage[many]{tcolorbox}    % boxes around definitions and theorems
%\usepackage{amsmath}
%\usepackage{nccmath}
\usepackage{amssymb}    % \ltimes
\usepackage{etoolbox}   % for start of Chapter
%\usepackage{amsfonts}
\usepackage{physics}    % for all Physics related
%\usepackage{dsfont}
%\usepackage{mathrsfs}

\usepackage{titling}
\usepackage{indentfirst}

\usepackage{bm}
\usepackage[dvipsnames]{xcolor}
\usepackage{cancel}

\usepackage{xurl}
\usepackage[colorlinks=true]{hyperref}

\usepackage{float}
\usepackage{graphicx}
\usepackage{subcaption}
%\usepackage{tikz}

\usepackage{ctable}     % tabelas
\renewcommand{\P}{\phantom{+}}  % empty space to indent things
\usepackage{multirow}
\usepackage{tabulary}

%%%%%%%%%%%%%%%%%%%%%%%%%%%%%%%%%%%%%%%%%%%%%%%%%%%

\newcommand{\eps}{\epsilon}
\newcommand{\vphi}{\varphi}
\newcommand{\cte}{\text{cte}}

\newcommand{\N}{{\mathbb{N}}}
\newcommand{\Z}{{\mathbb{Z}}}
%\newcommand{\Q}{{\mathbb{Q}}}
\newcommand{\C}{{\mathbb{C}}}
\renewcommand{\S}{{\hat{S}}}
%\renewcommand{\H}{\s{H}}

\renewcommand{\a}{{\vb{a}}}
\renewcommand{\b}{{\vb{b}}}
\renewcommand{\d}{{\dagger}}
\newcommand{\up}{{\uparrow}}
\newcommand{\down}{{\downarrow}}
\newcommand{\hc}{{\text{h.c.}}}

\newcommand{\ihat}{\bm{\hat{\imath}}}
\newcommand{\jhat}{\bm{\hat{\jmath}}}
\newcommand{\khat}{\bm{\hat{k}}}

\newcommand{\0}{{\vb{0}}}
%\newcommand{\1}{\mathds{1}}
\newcommand{\E}{{\vb{E}}}
\newcommand{\B}{{\vb{B}}}
\renewcommand{\u}{{\vb{u}}}
\renewcommand{\v}{{\vb{v}}}
\renewcommand{\r}{{\vb{r}}}
\newcommand{\R}{{\vb{R}}}
\newcommand{\Q}{{\vb{Q}}}
\newcommand{\G}{{\vb{G}}}
\newcommand{\g}{{\vb{g}}}
\renewcommand{\k}{{\vb{k}}}
\newcommand{\K}{{\vb{K}}}
\newcommand{\p}{{\vb{p}}}
\newcommand{\q}{{\vb{q}}}
\newcommand{\F}{{\vb{F}}}
\renewcommand{\t}{{\vb{t}}}
\newcommand{\vtau}{{\bm{\tau}}}
\newcommand{\vdelta}{{\bm{\delta}}}

\newcommand{\s}{\sigma}
%\newcommand{\prodint}[2]{\left\langle #1 , #2 \right\rangle}
\newcommand{\cc}[1]{\overline{#1}}
\newcommand{\Eval}[3]{\eval{\left( #1 \right)}_{#2}^{#3}}
\newcommand{\sg}[2]{\{ #1 \mid #2 \}}

\newcommand{\unit}[1]{\; \mathrm{#1}}

\newcommand{\n}{\medskip}
\newcommand{\e}{\quad \mathrm{and} \quad}
\newcommand{\ou}{\quad \mathrm{or} \quad}
\newcommand{\virg}{\, , \;}
\newcommand{\ptodo}{\forall \,}
\renewcommand{\implies}{\; \Rightarrow \;}
%\newcommand{\eqname}[1]{\tag*{#1}} % Tag equation with name

\setlength{\droptitle}{-7em}

\makeatletter
\patchcmd{\chapter}{\if@openright\cleardoublepage\else\clearpage\fi}{}{}{}  % start 'Chapter' at the same page. needs package etoolbox
\makeatother

%% Theorems, definitions, proofs
\theoremstyle{definition}

\newtheorem{definition}{Definition}[section]
\tcolorboxenvironment{definition}{
  colback=blue!5!white,
  boxrule=0pt,
  boxsep=1pt,
  left=2pt,right=2pt,top=2pt,bottom=2pt,
  oversize=2pt,
  sharp corners,
  before skip=\topsep,
  after skip=\topsep,
}

\newtheorem{theorem}{Theorem}[section]
\tcolorboxenvironment{theorem}{
  colback=blue!5!white,
  boxrule=0pt,
  boxsep=1pt,
  left=2pt,right=2pt,top=2pt,bottom=2pt,
  oversize=2pt,
  sharp corners,
  before skip=\topsep,
  after skip=\topsep,
}

\begin{document}
%%%%%%%%%%%%%%%%%%%%%%%%%%%%%%% COMMENT THIS TO COMPILE main.tex %%%%%%%%%%%%%%%%%%%%%%%%%%%%%%%%


%%%%%%%%%%%%%%%%%%%%%%%%%%%%%%%%%%%%%%%%%%%%%%%%%%%%%%%%%%%%%%%%%%%%%%%%%%%%%%%%%%%%%%%%%%%%%%%%%%
\chapter{Topological Quantum Chemistry}
%%%%%%%%%%%%%%%%%%%%%%%%%%%%%%%%%%%%%%%%%%%%%%%%%%%%%%%%%%%%%%%%%%%%%%%%%%%%%%%%%%%%%%%%%%%%%%%%%%

In this chapter we will introduce the formalism of Topological Quantum Chemistry (TQC) \cite{topological_quantum_chemistry2017}. Among many things, this theory mathematically defines when a electronic band is topological, which is a concept intrinsically associated to the construction of Wannier orbitals for a tight-binding model. The application of these concepts play a major role to motivate and construct a fully symmetric interacting MATBG model.

Since this theory involves numerous mathematical definitions and concepts, we will consistently use the example of monolayer graphene, which belongs to wallpaper group \#17 and the corresponding space group $P6mm$ (\#183).

\section{Wyckoff positions}

We are following the references \cite{lectures_tms2017, building_blocks2018}.

\begin{definition}[Orbit of $\q$] \label{def:orbit_q}
It is the set of all positions related to $\q$ by elements of the space group $G$, i.e. $\text{Orb}_\q = \{g \q \mid g \in G\}$, \textit{and} belong to the same unit cell.
\end{definition}

\begin{definition}[Site-symmetry group / Stabilizer group] \label{def:sitesym}
The site-symmetry group of a position $\q$ is the subgroup of operations $g \in G$ that leave $\q$ fixed.
$$
G_\q = \{g \mid g \q = \q\} \leq G.
$$
\end{definition}

\textit{Remarks:}
\begin{itemize}
\item $G_\q$ can include elements $\{R \mid \r\}$, with nonzero translations, $\r \neq \0$.
\item Since any site-symmetry group leaves a point invariant, it is also isomorphic to one of the 32 crystallographic points groups (in 3D).
\end{itemize}

\begin{definition}[Wyckoff position] \label{def:wyckpos}
A \textit{Wyckoff position} $\q$ is any position in the unit cell of the crystal. There are \textit{special} Wyckoff positions, which are those which are left invariant by some symmetry operations. The \textit{multiplicity} of a Wyckoff positions is the number of elements in its orbit (in the same unit cell).
\end{definition}

\begin{definition}[Coset representatives] \label{def:cosetrep}
The \textit{coset representatives} of site-symmetry group are the elements that generate the orbit of Wyckoff position. The number of coset representatives is also equal to the multiplicity of the Wyckoff position.
\end{definition}

\begin{definition}[Coset decomposition] \label{def:cosetdecomp}
Given a Wyckoff position $\q$, the \textit{coset decomposition} with respect to $\q$ is the full space group is defined by
$$
G = \bigcup_\alpha g_\alpha (G_{\q} \ltimes \Z^d).
$$
\end{definition}

\begin{definition}[Maximal Wyckoff position]
A site-symmetry group is \textbf{non-maximal} if there exists a finite group $H \neq G_\q$, such that $G_\q \subseteq H \subseteq G$. A site-symmetry group that is not non-maximal is \textbf{maximal}. A Wyckoff position containing $\q$ is maximal if the stabilizer group $G_\q$ is maximal.
\end{definition}

\section{Band Representations}

Let a site $\q$ belong to a Wyckoff position of multiplicity $n$, and suppose that $n_q$ orbitals reside on it. The wavefunctions $W_{i1}(\r)$, $i = 1, \ldots, n_q$, of these orbitals transform under an $n_q$-dimensional representation $\rho$, of the site-symmetry group $G_\q$:
$$
g W_{i1}(\r) = W_{i1}(g^{-1} \r) = [\rho(g)]_{ij} W_{j1}(\r).
$$

Without loss of generality, choose the equivalent sites $\q_\alpha = g_{\alpha} \q$ to be in the same unit cell as $\q$. Their orbitals transform under the conjugate representation $\rho_\alpha(h) = \rho(g_\alpha^{-1} h g_\alpha)$, where $h \in G_{\q_\alpha}$ and $g_\alpha^{-1} h g_\alpha \in G_\q$. The wavefunctions localized on $\q_\alpha$ are then defined by
$$
W_{i\alpha}(\r) = g_\alpha W_{i1}(\r) = W_{i1}(g_\alpha^{-1} \r),
$$
where $\alpha = 1, \ldots, n$ indexes the equivalent sites that belong to the Wyckoff position of multiplicity $n$. Now, the wavefunctions on other unit cells are obtained by an applied translation
$$
\sg{E}{\t_\mu} W_{i\alpha}(\r) = W_{i\alpha}(\r-\t_\mu),
$$
where $\t_\mu$ is a Bravais lattice vector.

Notice that we end up with $n \times n_\q \times N$ wavefunctions $W_{i\alpha}(\r-\t_\mu)$, where $N \to \infty$ is the number of unit cells in the system. Now we derive how they should transform under an arbitrary element $h = \sg{R}{\t} \in G$ of the space group. This will give us the band representation $\rho_G(h)$ induced from a representation $\rho$ of $G_\q$.
$$
\rho_G(h) W_{i\alpha}(\r-\t_\mu) =
\underbrace{h \sg{E}{\t_\mu}}_{= \sg{E}{R\t_\mu} h} W_{i\alpha} (\r) =
\sg{E}{R\t_\mu} h W_{i\alpha} (\r) =
\sg{E}{R\t_\mu} \, h g_\alpha \, W_{i1} (\r).
$$

From the coset decomposition of Definition \ref{def:cosetdecomp}, we write $h g_\alpha = \sg{E}{\t_{\beta\alpha}} g_\beta g$, for some $g \in G_\q$, coset representative $g_\beta$, and a Bravais lattice vector $\t_{\beta\alpha}$.

We can obtain a relation for the lattice vector $\t_{\beta\alpha}$. Notice that
$$
h g_\alpha \q = h \q_\alpha =
$$
$$
= \sg{E}{\t_{\beta\alpha}} g_\beta g \q = \sg{E}{\t_{\beta\alpha}} g_\beta \q = \sg{E}{\t_{\beta\alpha}} \q_\beta = \q_\beta + \t_{\beta\alpha}.
$$
This gives us that $\t_{\beta\alpha} = h \q_\alpha - \q_\beta$.

Therefore, $g$ and $\t_{\beta\alpha}$ are determined by
\begin{equation} \label{eq:g_and_tba}
h g_\alpha = \sg{E}{\t_{\beta\alpha}} g_\beta g, \quad \t_{\beta\alpha} = h \q_\alpha - \q_\beta.
\end{equation}

Following the derivation:
$$
\rho_G(h) W_{i\alpha}(\r-\t_\mu) =
\sg{E}{R\t_\mu} \sg{E}{\t_{\beta\alpha}} g_\beta g W_{i1}(\r) =
\sg{E}{R\t_\mu + \t_{\beta\alpha}} g_\beta [\rho(g)]_{ij} W_{j1}(\r) =
$$
$$
\sg{E}{R\t_\mu + \t_{\beta\alpha}} g_\beta [\rho(g)]_{ij} W_{j\beta}(\r) \implies
$$
\begin{equation} \label{eq:wannier_rep}
\rho_G(h) W_{i\alpha}(\r-\t_\mu) = [\rho(g)]_{ij} W_{j\beta}(\r - R\t_\mu - \t_{\beta\alpha}).
\end{equation}

\n

Now define the Fourier transformed Wannier functions (Bloch functions):
$$
a_{i\alpha}(\k, \r) = \sum_{\mu} e^{i\k\vdot\t_\mu} W_{i\alpha}(\r-\t_\mu).
$$

Using Equation \ref{eq:wannier_rep}, the Bloch functions transform as
$$
\rho_G(h) a_{i\alpha}(\k,\r) =
\rho_G(h) \sum_{\mu} e^{i\k\vdot\t_\mu} W_{i\alpha}(\r-\t_\mu) =
\sum_{\mu} e^{i\k\vdot\t_\mu} [\rho(g)]_{ij} W_{j\beta}(\r-R\t_\mu-\t_{\beta\alpha}) =
$$
$$
= e^{i(R\k)\vdot\t_{\beta\alpha}} [\rho(g)]_{ji} \sum_{\mu} e^{i(R\k)\vdot(R\t_\mu+\t_{\beta\alpha})} W_{j\beta}(\r-R\t_\mu-\t_{\beta\alpha}) \implies
$$
\begin{equation} \label{eq:bloch_rep}
(\rho_G(h) a)_{i\alpha}(\k,\r) = e^{i(R\k)\vdot\t_{\beta\alpha}} [\rho(g)]_{ji} \, a_{j\beta}(R\k, \r),
\end{equation}
where $g$ and $\t_{\beta\alpha}$ are determined by Equation \ref{eq:g_and_tba}. The choice of representatives $g_\alpha$ must be kept fixed through the construction.

Observe that $\rho_G(h)$ in its matrix form consists of infinitely many $(n\cdot n_q)\times (n\cdot n_q)$ block, where each one is labelled by a $(\k', \k)$ pair, that correspond to Bloch functions labelled by $\k', \k$. For a given $h = \sg{R}{\v} \in G$ and each set of columns corresponding to $\k$, there is exactly one non-zero block, identified by $\k' = R\k$. Denoting this block by $\rho_G^\k(h)$, its matrix elements are given by
$$
\rho_G^\k(h)_{j\beta,i\alpha} = e^{-i(R\k)\vdot\t_{\beta\alpha}}\rho_{ji}(g_\beta^{-1}\sg{E}{-\t_{\beta\alpha}}hg_\alpha).
$$

The full set of matrices $\rho_G^\k(h)$, for each $\k$ in the first BZ, contain all the non-zero elements of $\rho_G(h)$.

\section{Momentum space}

\begin{definition}[\textbf{Little group}]
Two reciprocal space vectors $\k_1$ and $\k_2$ are said to be equivalent, $\k_1 \equiv \k_2$, if $\k_2 - \k_1$ is a reciprocal lattice vector. The \textit{little group} $G_\k$ of a vector $\k$ in reciprocal space is the set of elements $g \in G$ such that $g \k \equiv \k$. Remember that the action of space group elements on reciprocal space is defined by
$$
g\k = \sg{R}{\t}\k = R\k.
$$
For each $\k$, notice that $G_\k$ is infinite because if $h \in G_\k$, the operation of $h$ followed by any Bravais lattice translation also belongs to $G_\k$.
\end{definition}

\n

The set $\{\rho_G^\k(h) \mid g \in G_\k\}$ furnishes an $(n\cdot n_q)\times(n\cdot n_q)$ representation of the little group $G_\k$, which we denote by $\rho_G \downarrow G_\k$; this is a subduction of $\rho_G$ onto $G_\k$, projected onto the Wannier functions at $\k$. Although $G_\k$ is infinite, the representaiton of two space group operations, $\sg{R}{\v}$ and $\sg{R}{\v+\t_1}$, where $\t_1$ is a Bravais lattice translation, will differ only by an overall phase $e^{-i(R\k)\vdot\t_1} = -e^{i\k\vdot\t_1}$ in $\rho_G \downarrow G_\k$.

\n

The characters of $\rho_G \downarrow G_\k$ are given by, for $h \in G_\k$,
$$
\rho_G^\k(h) =
\sum_{\alpha} e^{-i(R\k)\vdot\t_{\alpha\alpha}}
\tilde{\chi}[\rho(g_\alpha^{-1}\sg{E}{-\t_{\alpha\alpha}}h g_\alpha)],
$$
where
\begin{align} \label{eq:tilde_chi}
\tilde{\chi}[\rho(g)] =
\begin{cases}
\; \chi[\rho(g)], & \text{if } g \in G_\q \\
\; 0, & \text{if } g \notin G_\q
\end{cases}
\end{align}
and $\chi[\rho(g)]$ denotes the character of $g$ in the representation $\rho$.

\n

We would like to know how many times, $m_i^\k$, each irrep $\sigma_i^\k$ of $G_\k$ appears in $\rho_G \downarrow G_\k$:
\begin{equation} \label{eq:induce_subduce}
(\rho \uparrow G) \downarrow G_\k \equiv \bigoplus_i m_i^\k \sigma_i^\k,
\end{equation}
where the symbol $\equiv$ in Equation \ref{eq:induce_subduce} denotes the equivalence of representations.

\section{Elementary Band Representations}

We call two band representations equivalent if they are, in some sense, topologically equivalent. In the same sense as a homotopy.

\begin{definition}[\textbf{Equivalence between band representations}] \label{def:equiv_bandrep}
Two band representations $\rho_G$ and $\sigma_G$ are equivalent iff there exists a unitary matrix-valued function $S(\k,t,g)$ smooth in $\k$ and continuous in $t$ such that, for all $g \in G$
\begin{enumerate}
\item $S(\k, t, g)$ defines a band representation according to Equation \ref{eq:bloch_rep} for all $t \in [0,1]$;
\item $S(\k, 0, g) = \rho_G^\k(g)$;
\item $S(\k, 1, g) = \sigma_G^\k(g)$.
\end{enumerate}
\end{definition}

\begin{definition}[\textbf{Elementary band representation}]
A band representation is called \textbf{composite} if it is equivalent to the direct sum of other band representations. A band representation that is not composite is called \textbf{elementary}.
\end{definition}

\begin{theorem}[\textbf{Properties of band representations}]
Some properties of band representations are:
\begin{enumerate}
\item Because induction commutes with direct sums
$$
(\rho_1 \oplus \rho_2) \uparrow G = (\rho_1 \uparrow G) \oplus (\rho_2 \uparrow G),
$$
reducible representations of $G_\q$ induce composite band representations.

\item Given subgroups $K \subset H \subset G$, and a representation $\rho$ of $K$, because induction is transitive it follows that
$$
(\rho \uparrow H) \uparrow G = \rho \uparrow G.
$$
From this we conclude that all EBRs can be induced from irreps of the maximal site symmetry groups.
\end{enumerate}
\end{theorem}

\section{Exceptions}

There are exceptions where an irrep of the site symmetry group of a maximal Wyckoff position induces a composite band representation.

\section{Topological Systems}

\begin{definition}[\textbf{Topological band}]
A set of bands are in the \textbf{atomic limit} of a space group if they can be induced from localized Wannier functions consistent with the crystallice symmetry of that space group. Otherwise, they are \textbf{topological}.
\end{definition}

Band representations describe a system in the atomic limit. Topological band must be groups of bands that satisfy the crystal symmetry in momentum space, but nevertheless do not transform as a band representation. In other words, they cannot be induced from localized Wannier orbitals that obey the crystal symmetry.

\begin{theorem} \label{th:topo_insul}
Any isolated set of bands that is not equivalent to a band representation (composite or elementary) gives a strong, weak, or crystalline topological insulator.
\end{theorem}


\section{How to determine if a set of bands is topological}

As said in \cite{building_blocks2018}: \textbf{ESSA SEÇÃO EU COPIEI}

A practical route to determining whether a set of bands $\mathcal{B}$ is \textbf{not} a band representation is as follows: first, enumerate all EBRs for the particular space group and list the irreps that appear in each EBR at each high symmetry point. Next, compute the irreps at each high-symmetry point for the bands in $\mathcal{B}$. If the set of irreps that have been computed for the bands in $\mathcal{B}$ cannot be obtained from a linear combination of the EBRs in the space group, then the bands in $\mathcal{B}$ do not comprise a band representation and, by Theorem \ref{th:topo_insul}, are topological.

If the irreps that appear in $\mathcal{B}$ can be obtained from a linear combination of the EBRs of the space group, then one must compute symmetric and localized Wannier functions for the bands in $\mathcal{B}$ to confirm that they are equivalent to the atomic limit defined by the linear combination of EBRs or compute a Berry phase that will distinguish the two. This is because, it is possible for two distinct groups of bands to have the exact same irreps at all high-symmetry points, but different Berry phases (recall, this is exactly why we require the homotopic notion of equivalence, as in Definition \ref{def:equiv_bandrep}.)


\n\n

\textbf{REFINAR ESSE TEXTO COM BASE NO ARTIGO PRINCIPAL TOPOLOGICAL QUANTUM CHEMISTRY.}

%%%%%%%%%%%%%%%%%%%%%%%%%%%%%%%%%%%%%%%%%%%%%%%%%%%%%%%%%%%%%%%%%%%%%%%%%%%%%%%%%%%%%%%%%%%%%%%%%%
%%%%%%%%%%%%%%%%%%%%%%%%%%%%%%%%%%%%%%%%%%%%%%%%%%%%%%%%%%%%%%%%%%%%%%%%%%%%%%%%%%%%%%%%%%%%%%%%%%


%%%%%%%%%%%%%%%%%%%%%%%%%%%%%%% COMMENT THIS TO COMPILE main.tex %%%%%%%%%%%%%%%%%%%%%%%%%%%%%%%%
%%-----
%% Referências bibliográficas
%%-----
\addcontentsline{toc}{chapter}{\bibname}
%\bibliographystyle{abntex2-num}
\bibliography{citations}
\bibliographystyle{ieeetr}
\end{document}
%%%%%%%%%%%%%%%%%%%%%%%%%%%%%%% COMMENT THIS TO COMPILE main.tex %%%%%%%%%%%%%%%%%%%%%%%%%%%%%%%%
