%%%%%%%%%%%%%%%%%%%%%%%%%%%%%%% COMMENT THIS TO COMPILE main.tex %%%%%%%%%%%%%%%%%%%%%%%%%%%%%%%%
\documentclass[a4paper,12pt]{report}
\usepackage[english]{babel}
\usepackage[left=2cm,right=2cm,top=2cm,bottom=2cm]{geometry}
%\usepackage{mathtools}
\usepackage{amsthm}     % for definitions and theorems
\usepackage[many]{tcolorbox}    % boxes around definitions and theorems
%\usepackage{amsmath}
%\usepackage{nccmath}
\usepackage{amssymb}    % \ltimes
\usepackage{etoolbox}   % for start of Chapter
%\usepackage{amsfonts}
\usepackage{physics}    % for all Physics related
%\usepackage{dsfont}
%\usepackage{mathrsfs}

\usepackage{titling}
\usepackage{indentfirst}

\usepackage{bm}
\usepackage[dvipsnames]{xcolor}
\usepackage{cancel}

\usepackage{xurl}
\usepackage[colorlinks=true]{hyperref}

\usepackage{float}
\usepackage{graphicx}
\usepackage{subcaption}
%\usepackage{tikz}

\usepackage{ctable}     % tabelas
\renewcommand{\P}{\phantom{+}}  % empty space to indent things
\usepackage{multirow}
\usepackage{tabulary}

%%%%%%%%%%%%%%%%%%%%%%%%%%%%%%%%%%%%%%%%%%%%%%%%%%%

\newcommand{\eps}{\epsilon}
\newcommand{\vphi}{\varphi}
\newcommand{\cte}{\text{cte}}

\newcommand{\N}{{\mathbb{N}}}
\newcommand{\Z}{{\mathbb{Z}}}
%\newcommand{\Q}{{\mathbb{Q}}}
\newcommand{\C}{{\mathbb{C}}}
\renewcommand{\S}{{\hat{S}}}
%\renewcommand{\H}{\s{H}}

\renewcommand{\a}{{\vb{a}}}
\renewcommand{\b}{{\vb{b}}}
\renewcommand{\d}{{\dagger}}
\newcommand{\up}{{\uparrow}}
\newcommand{\down}{{\downarrow}}
\newcommand{\hc}{{\text{h.c.}}}

\newcommand{\ihat}{\bm{\hat{\imath}}}
\newcommand{\jhat}{\bm{\hat{\jmath}}}
\newcommand{\khat}{\bm{\hat{k}}}

\newcommand{\0}{{\vb{0}}}
%\newcommand{\1}{\mathds{1}}
\newcommand{\E}{{\vb{E}}}
\newcommand{\B}{{\vb{B}}}
\renewcommand{\u}{{\vb{u}}}
\renewcommand{\v}{{\vb{v}}}
\renewcommand{\r}{{\vb{r}}}
\newcommand{\R}{{\vb{R}}}
\newcommand{\Q}{{\vb{Q}}}
\newcommand{\G}{{\vb{G}}}
\newcommand{\g}{{\vb{g}}}
\renewcommand{\k}{{\vb{k}}}
\newcommand{\K}{{\vb{K}}}
\newcommand{\p}{{\vb{p}}}
\newcommand{\q}{{\vb{q}}}
\newcommand{\F}{{\vb{F}}}
\renewcommand{\t}{{\vb{t}}}
\newcommand{\vtau}{{\bm{\tau}}}
\newcommand{\vdelta}{{\bm{\delta}}}

\newcommand{\s}{\sigma}
%\newcommand{\prodint}[2]{\left\langle #1 , #2 \right\rangle}
\newcommand{\cc}[1]{\overline{#1}}
\newcommand{\Eval}[3]{\eval{\left( #1 \right)}_{#2}^{#3}}
\newcommand{\sg}[2]{\{ #1 \mid #2 \}}

\newcommand{\unit}[1]{\; \mathrm{#1}}

\newcommand{\n}{\medskip}
\newcommand{\e}{\quad \mathrm{and} \quad}
\newcommand{\ou}{\quad \mathrm{or} \quad}
\newcommand{\virg}{\, , \;}
\newcommand{\ptodo}{\forall \,}
\renewcommand{\implies}{\; \Rightarrow \;}
%\newcommand{\eqname}[1]{\tag*{#1}} % Tag equation with name

\setlength{\droptitle}{-7em}

\makeatletter
\patchcmd{\chapter}{\if@openright\cleardoublepage\else\clearpage\fi}{}{}{}  % start 'Chapter' at the same page. needs package etoolbox
\makeatother

%% Theorems, definitions, proofs
\theoremstyle{definition}

\newtheorem{definition}{Definition}[section]
\tcolorboxenvironment{definition}{
  colback=blue!5!white,
  boxrule=0pt,
  boxsep=1pt,
  left=2pt,right=2pt,top=2pt,bottom=2pt,
  oversize=2pt,
  sharp corners,
  before skip=\topsep,
  after skip=\topsep,
}

\newtheorem{theorem}{Theorem}[section]
\tcolorboxenvironment{theorem}{
  colback=blue!5!white,
  boxrule=0pt,
  boxsep=1pt,
  left=2pt,right=2pt,top=2pt,bottom=2pt,
  oversize=2pt,
  sharp corners,
  before skip=\topsep,
  after skip=\topsep,
}

\begin{document}
%%%%%%%%%%%%%%%%%%%%%%%%%%%%%%% COMMENT THIS TO COMPILE main.tex %%%%%%%%%%%%%%%%%%%%%%%%%%%%%%%%


%%%%%%%%%%%%%%%%%%%%%%%%%%%%%%%%%%%%%%%%%%%%%%%%%%%%%%%%%%%%%%%%%%%%%%%%%%%%%%%%%%%%%%%%%%%%%%%%%%
\chapter{Topological Quantum Chemistry}
%%%%%%%%%%%%%%%%%%%%%%%%%%%%%%%%%%%%%%%%%%%%%%%%%%%%%%%%%%%%%%%%%%%%%%%%%%%%%%%%%%%%%%%%%%%%%%%%%%

We are following the references \cite{lectures_tms2017, building_blocks2018}.

In this chapter we will introduce the formalism of Topological Quantum Chemistry (TQC) \cite{topological_quantum_chemistry2017}. Among many things, this theory mathematically defines when a electronic band is topological, which is a concept intrinsically associated to the construction of Wannier orbitals for a tight-binding model. The application of these concepts play a major role to motivate and construct a fully symmetric interacting MATBG model.

Since this theory involves numerous mathematical definitions and concepts, we will consistently use the example of monolayer graphene, which belongs to wallpaper group \#17 and the corresponding space group $P6mm$ ($\#183$ in international tables).
\begin{figure}[H]
\centering
\includegraphics[width=0.32\linewidth]{fig/honeycomb_C3.png}\hfill
\includegraphics[width=0.32\linewidth]{fig/honeycomb_C2.png}\hfill
\includegraphics[width=0.32\linewidth]{fig/honeycomb_sigma.png}
\caption{Generators of the point group \(\{C_3, C_2, m_{1\bar{1}}\}\) for the space group \(P6mm\) of the honeycomb lattice, with the coordinate system origin $O$ located at a hexagon's center.}
\label{fig:generators_P6mm}
\end{figure}

Following Figure \ref{fig:generators_P6mm}, we use the basis $\mathcal{B} = \{\a_1 = a \vu{x}, \a_2 = \frac{a}{2} \vu{x} + \frac{a\sqrt{3}}{2} \vu{y}\}$. In this basis, the generators \(\{C_3, C_2, m_{1\bar{1}}\}\) for the point group $6mm$ act like:
\begin{equation} \label{eq:generators_P6mm}
\begin{cases}
\; C_3 \a_1 = -\a_1+\a_2 \\
\; C_3 \a_2 = -\a_1
\end{cases}
\quad
\begin{cases}
\; C_2\a_1 = -\a_1 \\
\; C_2\a_2 = -\a_2
\end{cases}
\quad
\begin{cases}
\; m_{1\bar{1}} \a_1 = \a_2 \\
\; m_{1\bar{1}} \a_2 = \a_1
\end{cases}
\end{equation}

%%%%%%%%%%%%%%%%%%%%%%%%%%%%%%%%%%%%%%%%%%%%%%%%%%%%%%%%%%%%%%%%%%%%%%%%%%%%%%%%%%%%%%%%%%%%%%%%%%
\section{Definitions}
%%%%%%%%%%%%%%%%%%%%%%%%%%%%%%%%%%%%%%%%%%%%%%%%%%%%%%%%%%%%%%%%%%%%%%%%%%%%%%%%%%%%%%%%%%%%%%%%%%


\begin{definition}[\textbf{Orbit}] \label{def:orbit_q}
Given a point \(\q\), the \textit{orbit} of \(\q\) is the set of all points \textbf{within the same unit cell} that are related to \(\q\) by symmetry elements of the space group \(G\), i.e.,
\begin{equation} \label{eq:orbit_of_q}
\text{Orb}(\q) = G \q = \{g \q \mid g \in G\}.
\end{equation}
\end{definition}

\begin{example} \label{ex:orbit_1a2b3c}
In Figure \ref{fig:unitcell_q1q2q3q4q5}, using the basis $\mathcal{B} = \{\a_1, \a_2\}$, consider the points \(\textcolor{red}{\varhexagonblack} = \q_1 = (0,0)\), \(\textcolor{blue}{\mdblksquare} = \q_2 = \qty(\frac{1}{3}, \frac{1}{3})\), \(\textcolor{green}{\bigstar} = \q_3 = \qty(\frac{1}{2}, 0)\), \(\textcolor{magenta}{\varheartsuit} = \q_4 = \qty(\frac{1}{6}, \frac{1}{6})\), and \(\textcolor{orange}{\spadesuit} = \q_5 = \qty(\frac{1}{4}, 0)\).

Note that the unit cell, as shown in Figure \ref{fig:unitcell_limit}, excludes points that are equivalent under translations by Bravais lattice vectors. By applying all symmetry elements of \(G\) to the points \(\q_1, \q_2, \q_3, \q_4, \q_5\) and retaining only those that lie within the same unit cell, the resulting orbits are represented by the symbols \(\textcolor{red}{\varhexagonblack}\), \(\textcolor{blue}{\mdblksquare}\), \(\textcolor{green}{\bigstar}\), \(\textcolor{magenta}{\varheartsuit}\), and \(\textcolor{orange}{\spadesuit}\), as shown in Figure \ref{fig:unitcell_orbitsymbols}.
\end{example}

\begin{figure}[H]
\centering
\begin{subfigure}{.3\textwidth}
  \centering
  \includegraphics[width=\linewidth]{fig/unitcell_q1q2q3q4q5.png}
  \caption{Points $\q_1,\q_2,\q_3,\q_4,\q_5$}
  \label{fig:unitcell_q1q2q3q4q5}
\end{subfigure}
\hfill
\begin{subfigure}{.3\textwidth}
  \centering
  \includegraphics[width=\linewidth]{fig/unitcell_orbitsymbols.png}
  \caption{Orbits and Wyckoff letters}
  \label{fig:unitcell_orbitsymbols}
\end{subfigure}
\hfill
\begin{subfigure}{.235\textwidth}
  \centering
  \includegraphics[width=\linewidth]{fig/unitcell_limit.png}
  \caption{Unit cell choice}
  \label{fig:unitcell_limit}
\end{subfigure}
\caption{Unit cell and orbits.}
\label{fig:unitcell_orbits}
\end{figure}


\begin{definition}[\textbf{Site-symmetry group / Stabilizer group}] \label{def:sitesym}
The site-symmetry group of a position $\q$ is the subgroup of operations $g \in G$ that leave $\q$ fixed.
\begin{equation} \label{eq:site-symmetry_def}
G_\q = \{g \mid g \q = \q\} \subseteq G.
\end{equation}
\end{definition}

\textit{Remarks:}
\begin{itemize}
\item The site-symmetry group \( G_\q \) may include elements \(\{R \mid \r\}\) with \(\r \neq \0\).
\item Since any site-symmetry group leaves a point invariant, it is isomorphic to one of the crystallographic point groups: $32$ in three dimensions and $10$ in two dimensions.

\end{itemize}

\begin{example} \label{ex:site-symmetry_groups_p6mm}
Let us identify the site-symmetry groups of points $\q_1$, $\q_2$, $\q_3$, $\q_4$ and $\q_5$.

\begin{itemize}
\item The point $\textcolor{orange}{\spadesuit} = \q_5 = \qty(\frac{1}{4}, 0)$ has a site-symmetry group \(G_{\q_5}\) with only the identity $\{E \mid 0\}$ and $\{m_{x} \mid 0\}$ (reflection by the $x$-axis) as symmetry elements. This makes $G_{\q_5}$ isomorphic to $D_1$.
\begin{equation} \label{eq:site-symmetry-q5}
\{m_{x} \mid 0\} \q_3 = m_x \qty(\frac{1}{4} \a_1) = m_x \qty(\frac{a}{4} \vu{x}) = \frac{a}{4} \vu{x} = \frac{1}{4} \a_1 = \q_5.
\end{equation}


\item The point \(\textcolor{magenta}{\varheartsuit} = \q_4 = \qty(\frac{1}{6}, \frac{1}{6})\) has a site-symmetry group \(G_{\q_4}\) with only the identity $\{E \mid 0\}$ and $\{m_{1\cc{1}} \mid 0\}$ as symmetry elements. This also makes $G_{\q_4}$ isomorphic to $D_1$.
\begin{align} \label{eq:site-symmetry-q4}
\{m_{1\cc{1}} \mid 0\} \q_4 &= m_{1\cc{1}}\qty(\frac{1}{6} \a_1 + \frac{1}{6} \a_2) = \frac{1}{6} \a_2 + \frac{1}{6} \a_1 = \q_4.
\end{align}

\item The point \(\textcolor{green}{\bigstar} = \q_3 = \qty(\frac{1}{2}, 0)\) has a site-symmetry group \(G_{\q_3}\), which is isomorphic to \(D_2\). We can choose its generators to be $\{C_2 \mid \a_1\}$ and $\{m_{x}\mid 0\}$.
\begin{align} \label{eq:site-symmetry-q3}
\{C_2 \mid \a_1\} \q_3 &= C_2 \q_3 + \a_1 = -\q_3 + \a_1 = -\frac{1}{2} \a_1 + \a_1 = \q_3, \\
\{m_{x} \mid 0\} \q_3 &= m_x \qty(\frac{1}{2} \a_1) = m_x \qty(\frac{a}{2} \vu{x}) = \frac{1}{2} \a_1 = \q_3.
\end{align}

\item The point $\textcolor{blue}{\mdblksquare} = \q_2 = \qty(\frac{1}{3}, \frac{1}{3})$ has site-symmetry group $G_{\q_2}$ isomorphic to $D_3$, and its generators are $\{C_3 \mid \a_1\}$ and $\{m_{1\cc{1}} \mid 0\}$.
\begin{align} \label{eq:site-symmetry-q2}
\{C_3 \mid \a_1\} \q_2 &= C_3 \qty(\frac{1}{3} \a_1 + \frac{1}{3} \a_2) + \a_1 = \frac{1}{3} \qty(-2\a_1 + \a_2) + \a_1 =
\frac{1}{3} (\a_1 + \a_2) = \q_2, \\
\{m_{1\cc{1}} \mid 0\} \q_2 &= m_{1\cc{1}} \qty(\frac{1}{3} \a_1 + \frac{1}{3} \a_2) =
\frac{1}{3} \a_2 + \frac{1}{3} \a_1 = \q_2.
\end{align}

\item The point \(\textcolor{red}{\varhexagonblack} = \q_1 = (0,0)\) has a site-symmetry group \(G_{\q_1}\) isomorphic to \(D_6\), with generators \(\{C_3 \mid 0\}\), \(\{C_2 \mid 0\}\), and \(\{m_{1\cc{1}} \mid 0\}\). Since \(\q_1 = \0\) is the origin, it is evident that the action of any generator \(g\) on \(\q_1\) leaves it unchanged:
\begin{equation} \label{eq:site-symmetry-q1}
g \q_1 = \0 = \q_1, \quad \text{for all } g \in G_{\q_1}.
\end{equation}
\end{itemize}

\end{example}

\begin{definition}[\textbf{Wyckoff position}] \label{def:wyckpos}
A \textit{Wyckoff position} \( \q \) refers to any position within the unit cell of the crystal. Some Wyckoff positions are \textit{special}, meaning they are left invariant by certain symmetry operations (other than the identity $\{E \mid 0\}$). The \textit{multiplicity} $n$ of a Wyckoff position is the number of distinct positions in its orbit within the same unit cell, $n = \abs{\text{Orb}(q)}$. A Wyckoff position is generally denoted as \( n\ell \), where \( \ell \) is an alphabetical label that designates the increasing size of the site-symmetry group associated with the position.
\end{definition}

\begin{example} \label{ex:wyckpos_q1q2q3q4q5}
The Wyckoff positions \( \q_1 = \textcolor{red}{\varhexagonblack} \), \( \q_2 = \textcolor{blue}{\mdblksquare} \), \( \q_3 = \textcolor{green}{\bigstar} \), \( \q_4 = \textcolor{magenta}{\varheartsuit} \), and \( \q_5 = \textcolor{orange}{\spadesuit} \) are respectively labeled as \( \textcolor{red}{1a} \), \( \textcolor{blue}{2b} \), \( \textcolor{green}{3c} \), \( \textcolor{magenta}{6d} \), and \( \textcolor{orange}{6e} \), as shown in Figure \ref{fig:unitcell_orbitsymbols}.
\end{example}

\begin{definition}[\textbf{Coset representatives}] \label{def:cosetrep_wyckoff}
The \textit{coset representatives} of the site-symmetry group are the elements \( g_\alpha \) that generate the orbit of a Wyckoff position \( \q \), with each element in the orbit given by \( \q_\alpha = g_\alpha \q \). The number of coset representatives corresponds to the multiplicity of the Wyckoff position.
\end{definition}

\begin{example} \label{ex:coset_representatives_q1q2q3q4q5}
For the points $\q_1 = \textcolor{red}{\varhexagonblack}$, $\q_2 = \textcolor{blue}{\mdblksquare}$, $\q_3 = \textcolor{green}{\bigstar}$, $\q_4 = \textcolor{magenta}{\varheartsuit}$ and $\q_5 = \textcolor{orange}{\spadesuit}$, we can choose coset representatives as
\begin{align}
\text{Orb}(\q_1) &= \qty{\{E\mid 0\} \q_1}, \label{eq:orbit_q1} \\
\text{Orb}(\q_2) &= \qty{\{E\mid 0\} \q_2, \{C_6\mid 0\} \q_2}, \label{eq:orbit_q2} \\
\text{Orb}(\q_3) &= \qty{\{E\mid 0\} \q_3, \{C_6\mid 0\} \q_3, \{C_6^2\mid 0\} \q_3}, \label{eq:orbit_q3} \\
\text{Orb}(\q_4) &= \qty{\{E\mid 0\} \q_4, \{C_6\mid 0\} \q_4, \{C_6^2\mid 0\} \q_4, \{C_6^3\mid 0\} \q_4, \{C_6^4\mid 0\} \q_4, \{C_6^5\mid 0\} \q_4}, \label{eq:orbit_q4} \\
\text{Orb}(\q_5) &= \qty{\{E\mid 0\} \q_5, \{C_6\mid 0\} \q_5, \{C_6^2\mid 0\} \q_5, \{C_6^3\mid 0\} \q_5, \{C_6^4\mid 0\} \q_5, \{C_6^5\mid 0\} \q_5}, \label{eq:orbit_q5}
\end{align}
For example, in Equation \ref{eq:orbit_q3}, the coset representatives of $G_{\q_3}$ are $g_1 = \{E\mid 0\}$, $g_2 = \{C_6\mid 0\}$, and $g_3 = \{C_6^2\mid 0\}$.
\end{example}

\begin{lemma}[\textbf{Coset decomposition}] \label{lemma:cosetdecomp_wyckoff}
Let \( \q \) be a Wyckoff position in a space group \( G \). The coset decomposition of \( G \) with respect to \( \q \) is given by:
\begin{equation} \label{eq:coset_decomp_wyckoff}
G = \bigcup_\alpha g_\alpha (G_{\q} \ltimes \Z^d),
\end{equation}
where \( \Z^d \) represents the group of lattice translations, \(\{E\mid \t\}\), with \(\t = \sum_{i=1}^d n_i \a_i\), where \( n_i \in \Z \) and \(\{\a_i\}\) are the primitive lattice vectors in \( d \) dimensions.

In practice, this decomposition means that for any \( g \in G \), we can write \( g = \{E \mid \t\} g_\alpha h \), for some translation \(\t\), coset representative \( g_\alpha \), and \( h \in G_\q \). Applying \( \q \) to both sides:
\begin{equation} \label{eq:coset_decomp_translation_t_step1}
g \q = \{E \mid \t\} g_\alpha h \q = \{E \mid \t\} g_\alpha \q = \{E \mid \t\} \q_\alpha = \q_\alpha + \t,
\end{equation}
which leads to
\begin{equation} \label{eq:coset_decomp_translation_t}
\t = g \q - \q_\alpha.
\end{equation}
\end{lemma}

\begin{lemma}[\textbf{Conjugate site-symmetry groups}] \label{lemma:conjugate_group_G_qalpha}
Positions \(\q\) and \(\q_\alpha\) on the same orbit are connected by the relation \(\q_\alpha = g_\alpha \q\), where \(g_\alpha\) is a coset representative. For \(h \in G_\q\), the following holds:
\begin{equation} \label{eq:conjugate_group_qalpha}
h \q = \q \implies h g_\alpha^{-1} \q_\alpha = g_\alpha^{-1} \q_\alpha \implies
\qty(g_\alpha h g_\alpha^{-1}) \q_\alpha = \q_\alpha \implies
g_\alpha h g_\alpha^{-1} \in G_{\q_\alpha}.
\end{equation}
Consequently, the site-symmetry groups of points on the same orbit are conjugates of each other, satisfying \(G_{\q_\alpha} = g_\alpha G_\q g_\alpha^{-1}\).
\end{lemma}


\begin{definition}[\textbf{Maximal Wyckoff position}]
A site-symmetry group is \textbf{non-maximal} if there exists a finite group $H \neq G_\q$, such that $G_\q \subseteq H \subseteq G$. A site-symmetry group that is not non-maximal is \textbf{maximal}. A Wyckoff position containing $\q$ is maximal if the stabilizer group $G_\q$ is maximal.
\end{definition}

\section{Band Representations}

Let a site \(\q\) belong to a Wyckoff position of multiplicity \(n\), with \(n_q\) orbitals residing on it. The wavefunctions \(W_{i1}(\r)\), where \(i = 1, \ldots, n_q\), transform according to an \(n_q\)-dimensional representation \(\rho\) of the site-symmetry group \(G_\q\):
\begin{equation} \label{eq:gWi1_nq_rep}
h W_{i1}(\r) = W_{i1}(h^{-1} \r) = [\rho(h)]_{ij} W_{j1}(\r), \quad h \in G_\q.
\end{equation}

Without loss of generality, let the equivalent sites \(\q_\alpha = g_{\alpha} \q\) be in the same unit cell as \(\q\). The orbitals at \(\q_\alpha\) transform under the conjugate representation \(\rho_\alpha(h) = \rho(g_\alpha^{-1} h g_\alpha)\), where \(h \in G_{\q_\alpha}\) and \(g_\alpha^{-1} h g_\alpha \in G_\q\). The wavefunctions localized at \(\q_\alpha\) are then given by
\begin{equation} \label{eq:Wialpha_galpha_action}
W_{i\alpha}(\r) = g_\alpha W_{i1}(\r) = W_{i1}(g_\alpha^{-1} \r),
\end{equation}
where \(\alpha = 1, \ldots, n\) indexes the equivalent sites that belong to the Wyckoff position of multiplicity \(n\). The wavefunctions on other unit cells are obtained by applying a translation operator, such that:
\begin{equation} \label{eq:translated_Wialpha}
\sg{E}{\t_\mu} W_{i\alpha}(\r) = W_{i\alpha}(\r-\t_\mu),
\end{equation}
where $\t_\mu$ is a Bravais lattice vector.

We find that there are \(n \times n_\q \times N\) wavefunctions \(W_{i\alpha}(\r - \t_\mu)\), where \(N \to \infty\) represents the number of unit cells in the system. To determine how these wavefunctions transform under an arbitrary space group element \(h = \sg{R}{\t} \in G\), we derive the band representation \(\rho_G = \rho \uparrow G\) induced from the representation \(\rho\) of the site-symmetry group \(G_\q\).
\begin{equation} \label{eq:}
\rho_G(h) W_{i\alpha}(\r-\t_\mu) =
\underbrace{h \sg{E}{\t_\mu}}_{= \sg{E}{R\t_\mu} h} W_{i\alpha} (\r) =
\sg{E}{R\t_\mu} h W_{i\alpha} (\r) =
\sg{E}{R\t_\mu} \, h g_\alpha \, W_{i1} (\r).
\end{equation}

From the coset decomposition of Definition \ref{lemma:cosetdecomp_wyckoff}, we write $h g_\alpha = \sg{E}{\t_{\beta\alpha}} g_\beta g$, for some $g \in G_\q$, coset representative $g_\beta$, and a Bravais lattice vector $\t_{\beta\alpha}$.

Applying \ref{lemma:cosetdecomp_wyckoff}

We can obtain a relation for the lattice vector $\t_{\beta\alpha}$. Notice that
$$
h g_\alpha \q = h \q_\alpha =
$$
$$
= \sg{E}{\t_{\beta\alpha}} g_\beta g \q = \sg{E}{\t_{\beta\alpha}} g_\beta \q = \sg{E}{\t_{\beta\alpha}} \q_\beta = \q_\beta + \t_{\beta\alpha}.
$$
This gives us that $\t_{\beta\alpha} = h \q_\alpha - \q_\beta$.

Therefore, $g$ and $\t_{\beta\alpha}$ are determined by
\begin{equation} \label{eq:g_and_tba}
h g_\alpha = \sg{E}{\t_{\beta\alpha}} g_\beta g, \quad \t_{\beta\alpha} = h \q_\alpha - \q_\beta.
\end{equation}

Following the derivation:
$$
\rho_G(h) W_{i\alpha}(\r-\t_\mu) =
\sg{E}{R\t_\mu} \sg{E}{\t_{\beta\alpha}} g_\beta g W_{i1}(\r) =
\sg{E}{R\t_\mu + \t_{\beta\alpha}} g_\beta [\rho(g)]_{ij} W_{j1}(\r) =
$$
$$
\sg{E}{R\t_\mu + \t_{\beta\alpha}} g_\beta [\rho(g)]_{ij} W_{j\beta}(\r) \implies
$$
\begin{equation} \label{eq:wannier_rep}
\rho_G(h) W_{i\alpha}(\r-\t_\mu) = [\rho(g)]_{ij} W_{j\beta}(\r - R\t_\mu - \t_{\beta\alpha}).
\end{equation}

\n

Now define the Fourier transformed Wannier functions (Bloch functions):
$$
a_{i\alpha}(\k, \r) = \sum_{\mu} e^{i\k\vdot\t_\mu} W_{i\alpha}(\r-\t_\mu).
$$

Using Equation \ref{eq:wannier_rep}, the Bloch functions transform as
$$
\rho_G(h) a_{i\alpha}(\k,\r) =
\rho_G(h) \sum_{\mu} e^{i\k\vdot\t_\mu} W_{i\alpha}(\r-\t_\mu) =
\sum_{\mu} e^{i\k\vdot\t_\mu} [\rho(g)]_{ij} W_{j\beta}(\r-R\t_\mu-\t_{\beta\alpha}) =
$$
$$
= e^{i(R\k)\vdot\t_{\beta\alpha}} [\rho(g)]_{ji} \sum_{\mu} e^{i(R\k)\vdot(R\t_\mu+\t_{\beta\alpha})} W_{j\beta}(\r-R\t_\mu-\t_{\beta\alpha}) \implies
$$
\begin{equation} \label{eq:bloch_rep}
(\rho_G(h) a)_{i\alpha}(\k,\r) = e^{i(R\k)\vdot\t_{\beta\alpha}} [\rho(g)]_{ji} \, a_{j\beta}(R\k, \r),
\end{equation}
where $g$ and $\t_{\beta\alpha}$ are determined by Equation \ref{eq:g_and_tba}. The choice of representatives $g_\alpha$ must be kept fixed through the construction.

Observe that $\rho_G(h)$ in its matrix form consists of infinitely many $(n\cdot n_q)\times (n\cdot n_q)$ block, where each one is labelled by a $(\k', \k)$ pair, that correspond to Bloch functions labelled by $\k', \k$. For a given $h = \sg{R}{\v} \in G$ and each set of columns corresponding to $\k$, there is exactly one non-zero block, identified by $\k' = R\k$. Denoting this block by $\rho_G^\k(h)$, its matrix elements are given by
$$
\rho_G^\k(h)_{j\beta,i\alpha} = e^{-i(R\k)\vdot\t_{\beta\alpha}}\rho_{ji}(g_\beta^{-1}\sg{E}{-\t_{\beta\alpha}}hg_\alpha).
$$

The full set of matrices $\rho_G^\k(h)$, for each $\k$ in the first BZ, contain all the non-zero elements of $\rho_G(h)$.

\section{Momentum space}

\begin{definition}[\textbf{Little group}]
Two reciprocal space vectors $\k_1$ and $\k_2$ are said to be equivalent, $\k_1 \equiv \k_2$, if $\k_2 - \k_1$ is a reciprocal lattice vector. The \textit{little group} $G_\k$ of a vector $\k$ in reciprocal space is the set of elements $g \in G$ such that $g \k \equiv \k$. Remember that the action of space group elements on reciprocal space is defined by
$$
g\k = \sg{R}{\t}\k = R\k.
$$
For each $\k$, notice that $G_\k$ is infinite because if $h \in G_\k$, the operation of $h$ followed by any Bravais lattice translation also belongs to $G_\k$.
\end{definition}

\n

The set $\{\rho_G^\k(h) \mid g \in G_\k\}$ furnishes an $(n\cdot n_q)\times(n\cdot n_q)$ representation of the little group $G_\k$, which we denote by $\rho_G \downarrow G_\k$; this is a subduction of $\rho_G$ onto $G_\k$, projected onto the Wannier functions at $\k$. Although $G_\k$ is infinite, the representaiton of two space group operations, $\sg{R}{\v}$ and $\sg{R}{\v+\t_1}$, where $\t_1$ is a Bravais lattice translation, will differ only by an overall phase $e^{-i(R\k)\vdot\t_1} = -e^{i\k\vdot\t_1}$ in $\rho_G \downarrow G_\k$.

\n

The characters of $\rho_G \downarrow G_\k$ are given by, for $h \in G_\k$,
$$
\rho_G^\k(h) =
\sum_{\alpha} e^{-i(R\k)\vdot\t_{\alpha\alpha}}
\tilde{\chi}[\rho(g_\alpha^{-1}\sg{E}{-\t_{\alpha\alpha}}h g_\alpha)],
$$
where
\begin{align} \label{eq:tilde_chi}
\tilde{\chi}[\rho(g)] =
\begin{cases}
\; \chi[\rho(g)], & \text{if } g \in G_\q \\
\; 0, & \text{if } g \notin G_\q
\end{cases}
\end{align}
and $\chi[\rho(g)]$ denotes the character of $g$ in the representation $\rho$.

\n

We would like to know how many times, $m_i^\k$, each irrep $\sigma_i^\k$ of $G_\k$ appears in $\rho_G \downarrow G_\k$:
\begin{equation} \label{eq:induce_subduce}
(\rho \uparrow G) \downarrow G_\k \equiv \bigoplus_i m_i^\k \sigma_i^\k,
\end{equation}
where the symbol $\equiv$ in Equation \ref{eq:induce_subduce} denotes the equivalence of representations.

\section{Elementary Band Representations}

We call two band representations equivalent if they are, in some sense, topologically equivalent. In the same sense as a homotopy.

\begin{definition}[\textbf{Equivalence between band representations}] \label{def:equiv_bandrep}
Two band representations $\rho_G$ and $\sigma_G$ are equivalent iff there exists a unitary matrix-valued function $S(\k,t,g)$ smooth in $\k$ and continuous in $t$ such that, for all $g \in G$
\begin{enumerate}
\item $S(\k, t, g)$ defines a band representation according to Equation \ref{eq:bloch_rep} for all $t \in [0,1]$;
\item $S(\k, 0, g) = \rho_G^\k(g)$;
\item $S(\k, 1, g) = \sigma_G^\k(g)$.
\end{enumerate}
\end{definition}

\begin{definition}[\textbf{Elementary band representation}]
A band representation is called \textbf{composite} if it is equivalent to the direct sum of other band representations. A band representation that is not composite is called \textbf{elementary}.
\end{definition}

\begin{theorem}[\textbf{Properties of band representations}]
Some properties of band representations are:
\begin{enumerate}
\item Because induction commutes with direct sums
$$
(\rho_1 \oplus \rho_2) \uparrow G = (\rho_1 \uparrow G) \oplus (\rho_2 \uparrow G),
$$
reducible representations of $G_\q$ induce composite band representations.

\item Given subgroups $K \subset H \subset G$, and a representation $\rho$ of $K$, because induction is transitive it follows that
$$
(\rho \uparrow H) \uparrow G = \rho \uparrow G.
$$
From this we conclude that all EBRs can be induced from irreps of the maximal site symmetry groups.
\end{enumerate}
\end{theorem}

\section{Exceptions}

There are exceptions where an irrep of the site symmetry group of a maximal Wyckoff position induces a composite band representation.

\section{Topological Systems}

\begin{definition}[\textbf{Topological band}]
A set of bands are in the \textbf{atomic limit} of a space group if they can be induced from localized Wannier functions consistent with the crystallice symmetry of that space group. Otherwise, they are \textbf{topological}.
\end{definition}

Band representations describe a system in the atomic limit. Topological band must be groups of bands that satisfy the crystal symmetry in momentum space, but nevertheless do not transform as a band representation. In other words, they cannot be induced from localized Wannier orbitals that obey the crystal symmetry.

\begin{theorem} \label{th:topo_insul}
Any isolated set of bands that is not equivalent to a band representation (composite or elementary) gives a strong, weak, or crystalline topological insulator.
\end{theorem}


\section{How to determine if a set of bands is topological}

As said in \cite{building_blocks2018}: \textbf{ESSA SEÇÃO EU COPIEI}

A practical route to determining whether a set of bands $\mathcal{B}$ is \textbf{not} a band representation is as follows: first, enumerate all EBRs for the particular space group and list the irreps that appear in each EBR at each high symmetry point. Next, compute the irreps at each high-symmetry point for the bands in $\mathcal{B}$. If the set of irreps that have been computed for the bands in $\mathcal{B}$ cannot be obtained from a linear combination of the EBRs in the space group, then the bands in $\mathcal{B}$ do not comprise a band representation and, by Theorem \ref{th:topo_insul}, are topological.

If the irreps that appear in $\mathcal{B}$ can be obtained from a linear combination of the EBRs of the space group, then one must compute symmetric and localized Wannier functions for the bands in $\mathcal{B}$ to confirm that they are equivalent to the atomic limit defined by the linear combination of EBRs or compute a Berry phase that will distinguish the two. This is because, it is possible for two distinct groups of bands to have the exact same irreps at all high-symmetry points, but different Berry phases (recall, this is exactly why we require the homotopic notion of equivalence, as in Definition \ref{def:equiv_bandrep}.)


\n\n

\textbf{REFINAR ESSE TEXTO COM BASE NO ARTIGO PRINCIPAL TOPOLOGICAL QUANTUM CHEMISTRY.}

%%%%%%%%%%%%%%%%%%%%%%%%%%%%%%%%%%%%%%%%%%%%%%%%%%%%%%%%%%%%%%%%%%%%%%%%%%%%%%%%%%%%%%%%%%%%%%%%%%
%%%%%%%%%%%%%%%%%%%%%%%%%%%%%%%%%%%%%%%%%%%%%%%%%%%%%%%%%%%%%%%%%%%%%%%%%%%%%%%%%%%%%%%%%%%%%%%%%%


%%%%%%%%%%%%%%%%%%%%%%%%%%%%%%% COMMENT THIS TO COMPILE main.tex %%%%%%%%%%%%%%%%%%%%%%%%%%%%%%%%
%%-----
%% Referências bibliográficas
%%-----
\addcontentsline{toc}{chapter}{\bibname}
%\bibliographystyle{abntex2-num}
\bibliography{citations}
\bibliographystyle{ieeetr}
\end{document}
%%%%%%%%%%%%%%%%%%%%%%%%%%%%%%% COMMENT THIS TO COMPILE main.tex %%%%%%%%%%%%%%%%%%%%%%%%%%%%%%%%
