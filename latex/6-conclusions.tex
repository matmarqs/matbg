%%%%%%%%%%%%%%%%%%%%%%%%%%%%%%%% COMMENT THIS TO COMPILE main.tex %%%%%%%%%%%%%%%%%%%%%%%%%%%%%%%%
%\documentclass[a4paper,12pt]{report}
\usepackage[english]{babel}
\usepackage[left=2cm,right=2cm,top=2cm,bottom=2cm]{geometry}
%\usepackage{mathtools}
\usepackage{amsthm}     % for definitions and theorems
\usepackage[many]{tcolorbox}    % boxes around definitions and theorems
%\usepackage{amsmath}
%\usepackage{nccmath}
\usepackage{amssymb}    % \ltimes
\usepackage{etoolbox}   % for start of Chapter
%\usepackage{amsfonts}
\usepackage{physics}    % for all Physics related
%\usepackage{dsfont}
%\usepackage{mathrsfs}

\usepackage{titling}
\usepackage{indentfirst}

\usepackage{bm}
\usepackage[dvipsnames]{xcolor}
\usepackage{cancel}

\usepackage{xurl}
\usepackage[colorlinks=true]{hyperref}

\usepackage{float}
\usepackage{graphicx}
\usepackage{subcaption}
%\usepackage{tikz}

\usepackage{ctable}     % tabelas
\renewcommand{\P}{\phantom{+}}  % empty space to indent things
\usepackage{multirow}
\usepackage{tabulary}

%%%%%%%%%%%%%%%%%%%%%%%%%%%%%%%%%%%%%%%%%%%%%%%%%%%

\newcommand{\eps}{\epsilon}
\newcommand{\vphi}{\varphi}
\newcommand{\cte}{\text{cte}}

\newcommand{\N}{{\mathbb{N}}}
\newcommand{\Z}{{\mathbb{Z}}}
%\newcommand{\Q}{{\mathbb{Q}}}
\newcommand{\C}{{\mathbb{C}}}
\renewcommand{\S}{{\hat{S}}}
%\renewcommand{\H}{\s{H}}

\renewcommand{\a}{{\vb{a}}}
\renewcommand{\b}{{\vb{b}}}
\renewcommand{\d}{{\dagger}}
\newcommand{\up}{{\uparrow}}
\newcommand{\down}{{\downarrow}}
\newcommand{\hc}{{\text{h.c.}}}

\newcommand{\ihat}{\bm{\hat{\imath}}}
\newcommand{\jhat}{\bm{\hat{\jmath}}}
\newcommand{\khat}{\bm{\hat{k}}}

\newcommand{\0}{{\vb{0}}}
%\newcommand{\1}{\mathds{1}}
\newcommand{\E}{{\vb{E}}}
\newcommand{\B}{{\vb{B}}}
\renewcommand{\u}{{\vb{u}}}
\renewcommand{\v}{{\vb{v}}}
\renewcommand{\r}{{\vb{r}}}
\newcommand{\R}{{\vb{R}}}
\newcommand{\Q}{{\vb{Q}}}
\newcommand{\G}{{\vb{G}}}
\newcommand{\g}{{\vb{g}}}
\renewcommand{\k}{{\vb{k}}}
\newcommand{\K}{{\vb{K}}}
\newcommand{\p}{{\vb{p}}}
\newcommand{\q}{{\vb{q}}}
\newcommand{\F}{{\vb{F}}}
\renewcommand{\t}{{\vb{t}}}
\newcommand{\vtau}{{\bm{\tau}}}
\newcommand{\vdelta}{{\bm{\delta}}}

\newcommand{\s}{\sigma}
%\newcommand{\prodint}[2]{\left\langle #1 , #2 \right\rangle}
\newcommand{\cc}[1]{\overline{#1}}
\newcommand{\Eval}[3]{\eval{\left( #1 \right)}_{#2}^{#3}}
\newcommand{\sg}[2]{\{ #1 \mid #2 \}}

\newcommand{\unit}[1]{\; \mathrm{#1}}

\newcommand{\n}{\medskip}
\newcommand{\e}{\quad \mathrm{and} \quad}
\newcommand{\ou}{\quad \mathrm{or} \quad}
\newcommand{\virg}{\, , \;}
\newcommand{\ptodo}{\forall \,}
\renewcommand{\implies}{\; \Rightarrow \;}
%\newcommand{\eqname}[1]{\tag*{#1}} % Tag equation with name

\setlength{\droptitle}{-7em}

\makeatletter
\patchcmd{\chapter}{\if@openright\cleardoublepage\else\clearpage\fi}{}{}{}  % start 'Chapter' at the same page. needs package etoolbox
\makeatother

%% Theorems, definitions, proofs
\theoremstyle{definition}

\newtheorem{definition}{Definition}[section]
\tcolorboxenvironment{definition}{
  colback=blue!5!white,
  boxrule=0pt,
  boxsep=1pt,
  left=2pt,right=2pt,top=2pt,bottom=2pt,
  oversize=2pt,
  sharp corners,
  before skip=\topsep,
  after skip=\topsep,
}

\newtheorem{theorem}{Theorem}[section]
\tcolorboxenvironment{theorem}{
  colback=blue!5!white,
  boxrule=0pt,
  boxsep=1pt,
  left=2pt,right=2pt,top=2pt,bottom=2pt,
  oversize=2pt,
  sharp corners,
  before skip=\topsep,
  after skip=\topsep,
}

%\begin{document}
%%%%%%%%%%%%%%%%%%%%%%%%%%%%%%%% COMMENT THIS TO COMPILE main.tex %%%%%%%%%%%%%%%%%%%%%%%%%%%%%%%%


%%%%%%%%%%%%%%%%%%%%%%%%%%%%%%%%%%%%%%%%%%%%%%%%%%%%%%%%%%%%%%%%%%%%%%%%%%%%%%%%%%%%%%%%%%%%%%%%%%
\chapter{Conclusions} \label{ch:conclusions}
%%%%%%%%%%%%%%%%%%%%%%%%%%%%%%%%%%%%%%%%%%%%%%%%%%%%%%%%%%%%%%%%%%%%%%%%%%%%%%%%%%%%%%%%%%%%%%%%%%

We presented our research on TBG theory. We derived the Bistritzer-MacDonald model \cite{macdonald2011} in Section \ref{sec:BM-model}. Our intention is to utilize this knowledge to develop our own code for analyzing the electronic properties of the system. The BM model is advantageous as it is independent of whether a configuration of TBG is commensurate or not, and it also incorporates all the emergent symmetries observed in experiments \cite{zou2018}. In order to construct the Wannier orbitals localized on the AB and BA regions of interest, we recognized the necessity of thoroughly reviewing concepts of group theory in Chapter \ref{ch:group_theory}. Our studies will enable us to employ group theory to elucidate the Wannier Obstruction \cite{zou2018} and the Topological Heavy Fermion model \cite{topoheavyfermion2022}.

In this report, we began by exploring foundational concepts in group and representation theory, with a focus on the induction of representations and the definition of magnetic groups. These tools enable the reduction and construction of higher-dimensional representations by analyzing the symmetries of subgroups and their parent groups. We then introduced the framework of Topological Quantum Chemistry (TQC), which rigorously defines the topological properties of electronic bands by leveraging the symmetries of crystalline structures and the Wannier orbital formalism. This approach involves studying Wyckoff positions, coset decompositions, and the induction and subduction of representations to identify the irreducible components of bands, essential for understanding the topological implications of materials that deviate from the atomic limit. TQC provides a vital foundation for exploring topological phases in materials like topological insulators. The central focus of this report is on the Topological Heavy Fermion (THF) model, applied to Magic-Angle Twisted Bilayer Graphene (MATBG). By incorporating magnetic space group symmetries, emergent particle-hole symmetry, and the hybridization of higher-energy bands with flat bands, the THF model’s bandstructure (shown in Figure \ref{fig:thf-exploration}) aligns well with the BM model \cite{macdonald2011, topoheavyfermion2022} and offers a new avenue for insights into the superconducting properties of MATBG. The Wannier obstruction is resolved through Gaussian Wannier functions and the inclusion of higher-energy bands, which form the foundational basis of this model. This formalism paves the way for further investigations into the electronic properties of MATBG. Moving forward, the focus will be on completing the Master's thesis, with a work-in-progress version attached to this report, as we continue to advance our understanding of topological phenomena in twisted bilayer graphene.

We applied the TQC framework to analyze the MATBG system. For the BM model, the high-symmetry point representations of the bands are \( \Gamma_1 \oplus \Gamma_2 \), \( M_1 \oplus M_2 \), and \( K_2 K_3 \) \cite{all_magic_angles, bernevig_II_2021} at points \( \Gamma \), \( M \), and \( K \), respectively. Comparing these with the EBR tables for the space group \( P6'2'2 \) confirms that the system is topological and exhibits a Wannier obstruction. To address this issue, we studied the THF model \cite{topoheavyfermion2022}, which maps the low-energy bands of MATBG into a heavy fermion problem. This model resolves the Wannier obstruction by coupling \( f \)-electrons (flat bands, heavy) with \( c \)-electrons (nearly free), providing a complete description that includes all emergent symmetries.



%%%%%%%%%%%%%%%%%%%%%%%%%%%%%%%%%%%%%%%%%%%%%%%%%%%%%%%%%%%%%%%%%%%%%%%%%%%%%%%%%%%%%%%%%%%%%%%%%%
%%%%%%%%%%%%%%%%%%%%%%%%%%%%%%%%%%%%%%%%%%%%%%%%%%%%%%%%%%%%%%%%%%%%%%%%%%%%%%%%%%%%%%%%%%%%%%%%%%


%%%%%%%%%%%%%%%%%%%%%%%%%%%%%%%% COMMENT THIS TO COMPILE main.tex %%%%%%%%%%%%%%%%%%%%%%%%%%%%%%%%
%%%-----
%%% Referências bibliográficas
%%%-----
%\addcontentsline{toc}{chapter}{\bibname}
%%\bibliographystyle{abntex2-num}
%\bibliography{citations}
%\bibliographystyle{ieeetr}
%\end{document}
%%%%%%%%%%%%%%%%%%%%%%%%%%%%%%%% COMMENT THIS TO COMPILE main.tex %%%%%%%%%%%%%%%%%%%%%%%%%%%%%%%%
